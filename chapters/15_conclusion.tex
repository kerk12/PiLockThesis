Μέσα από όλα τα προηγούμενα κεφάλαια αναδείχτηκε η διαδικασία του σχεδιασμού και της υλοποίησης του PiLock, και παράλληλα περιγράφηκε εξονυχιστικά ο τρόπος λειτουργίας του. Έπειτα από όλη την διαδικασία αυτή, εξάχθηκαν συγκεκριμένα συμπεράσματα ως προς τον σχεδιασμό τέτοιου είδους έργων, είτε έχουν να κάνουν με έξυπνα σπίτια/συσκευές, είτε με ενσωμάτωση διάφορων συσκευών στο διαδίκτυο προκειμένου να είναι δυνατόν να ελεγχθούν.

Πρώτο και κυριότερο συμπέρασμα είναι οτι τέτοιου είδους συστήματα πρέπει να λειτουργούν με κατάλληλα σχεδιασμένα πρότυπα και δικλείδες ασφαλείας, προκειμένου να αποφευχθούν επιθέσεις τρίτων, ειδικά αν πρόκειται για σύστημα εξουσιοδότησης πρόσβασης, όπως το PiLock.

Το PiLock, καθώς πρόκειται για λογισμικό ανοικτού κώδικα, κληρονομεί τα προτερήματα των λογισμικών αυτού του είδους, και του παρέχεται δυνατότητα ανάπτυξης λειτουργικότητας και διόρθωσης λαθών από τους ίδιους του τους χρήστες, καθώς επίσης και, όπως αναφέρθηκε στην ενότητα \ref{foss_benefits}, μπορεί εξαιτίας της ανοικτότητάς του να χρησιμοποιείται για οποιοδήποτε σκοπό, είτε από ιδιότες, είτε από εταιρίες, και αυτό χωρίς κανένα κόστος.

Αναλύθηκε επίσης η καινοτομία του PiLock στις φορετές συσκευές, καθώς, όπως αναφέρθηκε στην ενότητα \ref{pilock_innovation}, είναι ένα από τα πρώτα οικιακά συστήματα πρόσβασης παγκοσμίως που χρησιμοποιεί Android Wear Smartwarch προκειμένου να ξεκλειδώνεται η πόρτα, και έπειτα από ενδελεχή έρευνα, πιθανόν να είναι και το πιο εξελιγμένο, μη εμπορικό σύστημα πρόσβασης στην κατηγορία αυτή. Το χαμηλό του κόστος, τα προσιτά του υλικά και η εύκολή του εγκατάσταση το κάνουν εύχρηστο και ιδανικό για το μεγαλύτερο ποσοστό των χρηστών του.

Είναι σημαντικό καθ'όλη την διαδικασία της ανάπτυξης ενός λογισμικού να χρησιμοποιείται κάποιο λογισμικό διαχείρησης εκδόσεων, προκειμένου να διαχειρίζεται αποτελεσματικά ο νέος κώδικας ο οποίος θα εισαχθεί στο λογισμικό, αλλά και για να κρατείται απομονωμένος από τον ήδη ενεργό κώδικα. Στην ενότητα \ref{subsec:vc}, αναλύθηκαν τα συστήματα διαχείρησης εκδόσεων καθώς επίσης και οι κανόνες που επινοήθηκαν προκειμένου να είναι ασφαλής η ανάπτυξη του PiLock, και να γίνεται αποτελεσματική διαχείριση του κώδικά του. Στο Κεφάλαιο \ref{ch:ci} αναλύθηκε η αναγκαιότητα χρήσης συστημάτων Συνεχούς Ενσωμάτωσης, μέσω των οποίων επιτυγχάνεται αποτελεσματικός έλεγχος του κώδικα, καθώς επίσης και όσο το δυνατόν γρηγορότερη εύρεση λαθών και διόρθωσή τους.

Τέλος, στο Κεφάλαιο \ref{ch:future_expansion}, αναλύθηκαν διάφοροι τρόποι επέκτασης του PiLock προκειμένου να προστεθούν χρήσιμες λειτουργίες σε αυτό (όπως η λειτουργία ξεκλειδώματος επισκεπτών), να βελτιωθεί η ασφάλεια του (ξεκλείδωμα με χρήση δακτυλικού αποτυπώματος), να ενσωματωθεί με ήδη υπάρχοντα συστήματα έξυπνων σπιτιών, καθώς επίσης και με εταιρίες (ενσωμάτωση LDAP, AD DS). Μέσω αυτών των προσθηκών, θα συνεχίσει να εξελίσσεται και θα γίνει ακόμα πιο εύκολο στην χρήση και ασφαλές.