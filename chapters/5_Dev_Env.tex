Στο παρόν κεφάλαιο θα αναφερθούμε στα εργαλεία που χρησιμοποιήθηκαν κατά την ανάπτυξη του PiLock, καθώς επίσης και στον τρόπο διαχείρησης του έργου ανάπτυξης.

\section{Η σημασία χρήσης δωρεάν λογισμικού ανοικτού κώδικα κατά την ανάπτυξη του PiLock}
	Ως "Λογισμικό Ανοικτού Κώδικα" (Open Source Software) ορίζεται το λογισμικό του οποίου ο πηγαίος κώδικας είναι διαθέσιμος ελεύθερα προς το κοινό προκειμένου να μπορεί να τροποποιηθεί, να αναβαθμιστεί ή να μελετηθεί. Ο πηγαίος κώδικας, για τον απλό χρήστη είναι ένα τμήμα του λογισμικού που δεν έχει δει ποτέ. Σε έργα ανοικτού κώδικα, μπορεί ο οποιοσδήποτε να προτείνει διορθώσεις, αναβαθμίσεις ή προσθήκη χαρακτηριστικών\textsuperscript{\cite{FOSS_def}}. 

	Εξαιτίας αυτού του χαρακτηριστικού, και των αδειών που το υποστηρίζουν, το λογισμικό ανοικτού κώδικα μπορεί να χρησιμοποιηθεί για οποιοδήποτε σκοπό επιθυμεί ο χρήστης, χωρίς να περιορίζεται από κάποια άδεια χρήσης. Επίσης, παρέχει αυξημένη ασφάλεια, εφόσον μπορεί να δοκιμαστεί και να μελετηθεί από τους προγραμματιστές ο πηγαίος κώδικάς του. Στο λογισμικό κλειστού κώδικα, είναι αδύνατον να μελετηθεί και να τροποποιηθεί από τρίτους ο κώδικάς του, και κατ' επέκταση, εφόσον υπάρξει ένα κενό ασφαλείας θα πάρει συνήθως αρκετά περισσότερο χρόνο μέχρι να κυκλοφορήσει μια ενημέρωση ασφάλειας. Τέλος, εξαιτίας της ευελιξίας που παρέχει, το λογισμικό ανοικτού κώδικα μπορεί να τροποποιηθεί προκειμένου να μπορέσει καλύτερα να καλύψει τις ανάγκες του χρήστη\textsuperscript{\cite{FOSS_def}, \cite{FOSS_benefits}}.

	Μία υποκατηγορία του λογισμικού κώδικα είναι το \textbf{Δωρεάν Λογισμικό Ανοικτού Κώδικα (Free and Open Source Software, FOSS)}, το οποίο επιτρέπει στον χρήστη να το κατεβάσει και να το χρησιμοποιήσει χωρίς κάποιο κόστος. Το λογισμικό που χρησιμοποιήθηκε για την ανάπτυξη του PiLock ανήκει στην κατηγορία αυτή.