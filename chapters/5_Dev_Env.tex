Στο παρόν κεφάλαιο θα αναφερθούμε στα εργαλεία που χρησιμοποιήθηκαν κατά την ανάπτυξη του PiLock, καθώς επίσης και στον τρόπο διαχείρησης του έργου ανάπτυξης.

\section{Η σημασία χρήσης δωρεάν λογισμικού ανοικτού κώδικα κατά την ανάπτυξη του PiLock}
	Ως "Λογισμικό Ανοικτού Κώδικα" (Open Source Software) ορίζεται το λογισμικό του οποίου ο πηγαίος κώδικας είναι διαθέσιμος ελεύθερα προς το κοινό προκειμένου να μπορεί να τροποποιηθεί, να αναβαθμιστεί ή να μελετηθεί. Ο πηγαίος κώδικας, για τον απλό χρήστη είναι ένα τμήμα του λογισμικού που δεν έχει δει ποτέ. Σε έργα ανοικτού κώδικα, μπορεί ο οποιοσδήποτε να προτείνει διορθώσεις, αναβαθμίσεις ή προσθήκη χαρακτηριστικών\textsuperscript{\cite{FOSS_def}}. 

	Εξαιτίας αυτού του χαρακτηριστικού, και των αδειών που το υποστηρίζουν, το λογισμικό ανοικτού κώδικα μπορεί να χρησιμοποιηθεί για οποιοδήποτε σκοπό επιθυμεί ο χρήστης, χωρίς να περιορίζεται από κάποια άδεια χρήσης. Επίσης, παρέχει αυξημένη ασφάλεια, εφόσον μπορεί να δοκιμαστεί και να μελετηθεί από τους προγραμματιστές ο πηγαίος κώδικάς του. Στο λογισμικό κλειστού κώδικα, είναι αδύνατον να μελετηθεί και να τροποποιηθεί από τρίτους ο κώδικάς του, και κατ' επέκταση, εφόσον υπάρξει ένα κενό ασφαλείας θα πάρει συνήθως αρκετά περισσότερο χρόνο μέχρι να κυκλοφορήσει μια ενημέρωση ασφάλειας. Τέλος, εξαιτίας της ευελιξίας που παρέχει, το λογισμικό ανοικτού κώδικα μπορεί να τροποποιηθεί προκειμένου να μπορέσει καλύτερα να καλύψει τις ανάγκες του χρήστη\textsuperscript{\cite{FOSS_def}, \cite{FOSS_benefits}}.

	Μία υποκατηγορία του λογισμικού ανοικτού κώδικα είναι το \textbf{Δωρεάν Λογισμικό Ανοικτού Κώδικα (Free and Open Source Software, FOSS)}, το οποίο επιτρέπει στον χρήστη να το κατεβάσει και να το χρησιμοποιήσει χωρίς κάποιο κόστος. Το λογισμικό που χρησιμοποιήθηκε για την ανάπτυξη του PiLock ανήκει στην κατηγορία αυτή.

\section{Version Control}
	Καθ' όλη την διαδικασία ανάπτυξης του PiLock χρησιμοποιήθηκε σύστημα Version Control. Τα Συστήματα Version Control βοηθούν τον/τους προγραμματιστή/ές καθώς προσδίδουν ασφάλεια και ευελιξία κατά την ανάπτυξη και την συντήρηση ενός έργου. To Git είναι το λογισμικό Version Control που χρησιμοποιήθηκε κατά την ανάπτυξη του PiLock.

	Το Git δημιουργήθηκε από τον Linus Torvalds το 2005 προκειμένου να τον βοηθήσει στην ανάπτυξη του Linux Kernel\textsuperscript{\cite{Git_History}}. Διάφοροι άλλοι συνεισφέροντες προς το Linux Kernel βοήθησαν στην ανάπτυξη του Git, κατά την πρώτη περίοδο της ανάπτυξής του. Αποτελεί δωρεάν λογισμικό ανοικτού κώδικα και διανείμεται υπό την άδεια GNU General Public Licence, έκδοση 2\textsuperscript{\cite{Git_Licence}}. 

	Πιο συγκεκριμένα, εξαιτίας της ικανότητάς του Git να διατηρεί ιστορικό για όλα τα αρχεία ενός έργου, το έργο μπορεί να επανέλθει σε μία προηγούμενη κατάστασή του, ανά πάσα στιγμή. Οι "καταστάσεις" είναι γνωστές ως "Commits". Με αυτό τον τρόπο, δεν απορρίπτεται κώδικας, πολλές φορές πολύτιμος για την ανάπτυξη ενός έργου. Μέσω των commits, μπορεί κάποιος να βεβαιωθεί ποιος έκανε αλλαγές στον κώδικα σε οποιοδήποτε σημείο επιθυμεί. Αξίζει να αναφερθεί η δυνατότητα της ψηφιακής υπογραφής των commits μέσω του GNU Privacy Guard (GPG), προκειμένου να αποφευχθεί προσωποποίηση\textsuperscript{\cite{Git_commit_assurance}}. Επίσης, μέσω του σύστημα staging του Git, ο προγραμματιστής γνωρίζει τι ακριβώς κρατείται σε ένα νέο Commit, όποτε καταχωρηθεί. Με το branching system του Git, καθίσταται δυνατόν να τροποποιείται ή να προστίθεται κώδικας και νέα features χωρίς να τροποποιείται ο Stable κώδικας του έργου (για παράδειγμα, το κάθε release χρησιμοποιεί υποχρεωτικά νέο branch), ο οποίος "ενημερώνεται" στο τέλος του κάθε release/feature, προκειμένου να αποφευχθούν προβλήματα. Τέλος, το Git διευκολύνει σημαντικά την συνεργασία μεταξύ προγραμματιστών καθώς μπορεί ο κάθε προγραμματιστής να δουλεύει το δικό του "κομμάτι" κώδικα, χωρίς να επηρρεάζει την πρόοδο των υπολοίπων προγραμματιστών που δουλεύουν πάνω στο έργο.

	Στην ανάπτυξη του PiLock, το λογισμικό της πλευράς του Εξυπηρετητή, του Πελάτη καθώς επίσης και τα σενάρια ξεκλειδώματος (Unlock Scripts) διαχειρίζονται ξεχωριστά σε διαφορετικά αποθετήρια. Συγκεκριμένα, τα σενάρια ξεκλειδώματος αποτελούν submodule του λογισμικού του εξυπηρετητή.

	Προκειμένου να γίνει σωστή διαχείρηση της διαδικασίας ανάπτυξης, ακολουθήθηκαν κάποιοι κανόνες. Οι κανόνες αυτοί διασφαλίζουν την ακαιρεότητα του κώδικα ανα πάσα στιγμή κατά την ανάπτυξη. Πιο συγκεκριμένα:

	\begin{itemize}
		\item Η σταθερή (Stable) έκδοση του κώδικα βρίσκεται στο master branch.
		\item Όποια αλλαγή πρόκειται να γίνει στον κώδικα, είτε αυτή είναι hotfix, είτε κάποιο νέο feature, είτε documentation, θα πρέπει να γίνεται αποκλειστικά σε νέο branch με χαρακτηριστικό τίτλο, ο οποίος να ξεκινά από το ανάλογο πρόθεμα (prefix). Συγκεκριμένα, αν πρόκειται για νέο feature να χρησιμοποιεί το "feature" prefix, αν πρόκειται για bugfix να χρησιμοποιεί το "bug" ή το "hotfix" prefix, και αν πρόκειται για αλλαγή στο documentation ή στο Readme, να χρησιμοποιεί το "doc" prefix. (Πχ. feature/pin\_changing)
		\item Τα νέα Branches, εφόσον ελεγχθούν, θα πρέπει να συγχωνεύονται στο αντίστοιχο branch στο οποίο απευθύνονται (βλ. παρακάτω).
		\item Νέα λειτουργικότητα (νέα features) προστίθεται μόνο στα νέα releases (στο branch του εκάστοτε νέου release). Τα hotfix καθώς επίσης και διάφορα bugs μπορούν να συγχωνεύονται απευθείας με το master, αρκεί να έχουν ελεγχθεί εξονυχιστικά πρώτα και να είναι υψηλής προτεραιότητας.
		\item Το master branch, καθώς επίσης και τα branches για νέα releases θα πρέπει να είναι κλειδωμένα και να δέχονται συγχονεύσεις μόνο εφόσον περάσει από έγκριση ο κώδικας μέσω κάποιου Pull Request ή Merge Request.
	\end{itemize} 

	Αρχικά, μέχρι την πρώτη δημόσια έκδοσή του (\verb|0.2.0|), ο κώδικας του PiLock φιλοξενούνταν στον προσωπικό εξυπηρετητή του δημιουργού, και διαχειρίζονταν τα αποθετήρια του μέσω του δωρεάν λογισμικού διαχείρησης αποθετηρίων Git γνωστό ως GitLab (\url{https://about.gitlab.com}). Αργότερα, από την πρώτη δημόσια έκδοσή του PiLock και μετά, ξεκίνησε να χρησιμοποιείται το GitHub (\url{https://github.com}) ως χώρος φιλοξενίας του έργου και των αποθετηρίων του.

\section{Issue Tracking}
	Κατά την ανάπτυξη του PiLock, έγινε χρήση τόσο του συστήματος Issue Tracking του GitLab, όσο και του GitHub. Τα συστήματα Issue Tracking, χρησιμοποιούνται για να κρατάνε μια λίστα με διάφορα "ζητήματα" που προκύπτουν κατά την ανάπτυξη ενός έργου ή που προέρχονται από εξωτερικούς χρήστες. Στην 2η κατηγορία ανήκουν διάφορα αιτήματα νέας λειτουργικότητας, ή διάφορα bugs τα οποία μπορεί να έχουν αναφερθεί, κατά την χρήση του λογισμικού ή κατά την διάρκεια δοκιμών (testing). Issues επίσης μπορεί να προκύψουν από εξωτερικούς χρήστες ως απλά ερωτήματα για την χρήση του λογισμικού.

	Τα Issues, έχουν 2 κύριες καταστάσεις: Open (Ανοικτό), Closed (Κλειστό/Ολοκληρωμένο). Τα ανοικτά issues είναι τα αυτά που ακόμα δεν έχουν ικανοποιηθεί οι απαιτήσεις τους, ή είναι σε διαδικασία ανάπτυξης. Τα κλειστά issues είναι τα εκπληρωμένα issues ή όσα issues δεν είναι δυνατόν να εκπληρωθούν για κάποιο λόγο, και δεν πρόκειται να αναπτυχθούν άλλο.

	Προκειμένου να μπορούν να κατηγοριοποιηθούν τα Issues ενός έργου, και να τα αναλάβουν, πολλές φορές διαφορετικές ομάδες, χρησιμοποιούνται οι ετικέτες (Labels). Κάποια χαρακτηριστικά παραδείγματα χρήσης ετικετών είναι για να χωριστούν τα issues που απαιτούν νέα λειτουργικότητα από τα issues που χρησιμοποιούνται προκειμένου να επιδιορθωθεί ένα bug.

	Όπως αναφέραμε στην αρχή, κατά την ανάπτυξη του PiLock, χρησιμοποιήθηκε Issue Tracking προκειμένου να οργανωθεί περισσότερο η διαδικασία της ανάπτυξης. Αξίζει να τονιστεί οτι έπειτα από την πρώτη δημόσια έκδοση του PiLock, μπορεί ο οποιοσδήποτε χρήστης του να συνεισφέρει στην συνεισφορά νέας λειτουργικότητας ή στην αναφορά και επίλυση bugs που υπάρχουν στο σύστημα.

	Το κάθε issue καταχωρείται στο αντίστοιχο milestone. Ως "Milestone" (Ορόσημο) ορίζεται ένα σημαντικό σημείο κατά την ανάπτυξη του έργου. Τα Milestones δημιουργούνται από τους συντηρητές ή τους διαχειριστές ενός έργου και χρησιμοποιούνται προκειμένου να οργανωθεί καλύτερα η ανάπτυξη του έργου και για να κατηγοριοποιηθούν τα issues. Για παράδειγμα, ως milestones συνήθως ορίζονται οι νέες εκδόσεις ενός έργου, πριν να γίνουν stable. Αφότου γίνουν Stable, το milestone κλείνει. 