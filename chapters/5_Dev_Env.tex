Στο παρόν κεφάλαιο θα αναφερθούν και θα αναλυθούν τα εργαλεία που χρησιμοποιήθηκαν κατά την ανάπτυξη του PiLock, καθώς επίσης και στην διαχείρηση του έργου ανάπτυξης.

\section{Η σημασία χρήσης δωρεάν λογισμικού ανοικτού κώδικα κατά την ανάπτυξη του PiLock}
	Ως "Λογισμικό Ανοικτού Κώδικα" (Open Source Software) ορίζεται το λογισμικό του οποίου ο πηγαίος κώδικας είναι διαθέσιμος ελεύθερα προς το κοινό προκειμένου να μπορεί να τροποποιηθεί, να αναβαθμιστεί ή να μελετηθεί. Ο πηγαίος κώδικας, για τον απλό χρήστη είναι ένα τμήμα του λογισμικού που δεν έχει δει ποτέ. Σε έργα ανοικτού κώδικα, μπορεί ο οποιοσδήποτε να προτείνει διορθώσεις, αναβαθμίσεις ή προσθήκη χαρακτηριστικών\textsuperscript{\cite{FOSS_def}}. 

	Εξαιτίας αυτού του χαρακτηριστικού, και των αδειών που το υποστηρίζουν, το λογισμικό ανοικτού κώδικα μπορεί να χρησιμοποιηθεί για οποιοδήποτε σκοπό επιθυμεί ο χρήστης, χωρίς να περιορίζεται από κάποια άδεια χρήσης. Επίσης, παρέχει αυξημένη ασφάλεια, εφόσον μπορεί να δοκιμαστεί και να μελετηθεί από τους προγραμματιστές ο πηγαίος κώδικάς του. Στο λογισμικό κλειστού κώδικα, είναι αδύνατον να μελετηθεί και να τροποποιηθεί από τρίτους ο κώδικάς του, και κατ' επέκταση, εφόσον υπάρξει ένα κενό ασφαλείας θα πάρει συνήθως αρκετά περισσότερο χρόνο μέχρι να κυκλοφορήσει μια ενημέρωση ασφάλειας. Τέλος, εξαιτίας της ευελιξίας που παρέχει, το λογισμικό ανοικτού κώδικα μπορεί να τροποποιηθεί προκειμένου να μπορέσει καλύτερα να καλύψει τις ανάγκες του χρήστη\textsuperscript{\cite{FOSS_def}, \cite{FOSS_benefits}}.

	Μία υποκατηγορία του λογισμικού ανοικτού κώδικα είναι το \textbf{Δωρεάν Λογισμικό Ανοικτού Κώδικα (Free and Open Source Software, FOSS)}, το οποίο επιτρέπει στον χρήστη να το κατεβάσει και να το χρησιμοποιήσει χωρίς κάποιο κόστος. Το λογισμικό που χρησιμοποιήθηκε για την ανάπτυξη του PiLock ανήκει στην κατηγορία αυτή.

\section{Διαχείρηση του έργου}

	\subsection{Version Control}
		\label{subsec:vc}
		Καθ' όλη την διαδικασία ανάπτυξης του PiLock χρησιμοποιήθηκε σύστημα Version Control. Τα Συστήματα Version Control βοηθούν τον/τους προγραμματιστή/ές καθώς προσδίδουν ασφάλεια και ευελιξία κατά την ανάπτυξη και την συντήρηση ενός έργου. To \idxa{Git} είναι το λογισμικό Version Control που χρησιμοποιήθηκε κατά την ανάπτυξη του PiLock.

		Το Git δημιουργήθηκε από τον Linus Torvalds το 2005 προκειμένου να τον βοηθήσει στην ανάπτυξη του Linux Kernel\textsuperscript{\cite{Git_History}}. Διάφοροι άλλοι συνεισφέροντες προς το Linux Kernel βοήθησαν στην ανάπτυξη του Git, κατά την πρώτη περίοδο της ανάπτυξής του. Αποτελεί δωρεάν λογισμικό ανοικτού κώδικα και διανείμεται υπό την άδεια GNU General Public Licence, έκδοση 2\textsuperscript{\cite{Git_Licence}}. 

		Πιο συγκεκριμένα, εξαιτίας της ικανότητάς του Git να διατηρεί ιστορικό για όλα τα αρχεία ενός έργου, το έργο μπορεί να επανέλθει σε μία προηγούμενη κατάστασή του, ανά πάσα στιγμή. Οι "καταστάσεις" είναι γνωστές ως "\idxa{Commits}". Με αυτό τον τρόπο, δεν απορρίπτεται κώδικας, πολλές φορές πολύτιμος για την ανάπτυξη ενός έργου. Μέσω των commits, μπορεί κάποιος να βεβαιωθεί ποιος έκανε αλλαγές στον κώδικα σε οποιοδήποτε σημείο επιθυμεί. Αξίζει να αναφερθεί η δυνατότητα της ψηφιακής υπογραφής των commits μέσω του \idxa{GNU Privacy Guard (GPG)}, προκειμένου να αποφευχθεί προσωποποίηση\textsuperscript{\cite{Git_commit_assurance}}. Επίσης, μέσω του συστήματος \idxa{Staging} του Git, ο προγραμματιστής γνωρίζει τι ακριβώς κρατείται σε ένα νέο Commit, όποτε καταχωρηθεί. Με το \idxa{Branching System} του Git, καθίσταται δυνατόν να τροποποιείται ή να προστίθεται κώδικας και νέα features χωρίς να τροποποιείται ο Stable κώδικας του έργου (για παράδειγμα, το κάθε release χρησιμοποιεί υποχρεωτικά νέο branch), ο οποίος "ενημερώνεται" στο τέλος του κάθε release/feature, προκειμένου να αποφευχθούν προβλήματα. Τέλος, το Git διευκολύνει σημαντικά την συνεργασία μεταξύ προγραμματιστών καθώς μπορεί ο κάθε προγραμματιστής να δουλεύει το δικό του "κομμάτι" κώδικα, χωρίς να επηρρεάζει την πρόοδο των υπολοίπων προγραμματιστών που δουλεύουν πάνω στο έργο.

		Στην ανάπτυξη του PiLock, το λογισμικό της πλευράς του Εξυπηρετητή, του Πελάτη καθώς επίσης και τα σενάρια ξεκλειδώματος (Unlock Scripts) διαχειρίζονται ξεχωριστά σε διαφορετικά αποθετήρια. Συγκεκριμένα, τα σενάρια ξεκλειδώματος αποτελούν Submodule του λογισμικού του εξυπηρετητή, δηλαδή μπορεί να εμφολευθεί ως ξεχωριστό αποθετήριο μέσα σε ένα ήδη υπάρχον αποθετήριο, ως κομμάτι του.

		Προκειμένου να γίνει σωστή διαχείρηση της διαδικασίας ανάπτυξης, ακολουθήθηκαν κάποιοι κανόνες. Οι κανόνες αυτοί διασφαλίζουν την ακαιρεότητα του κώδικα ανα πάσα στιγμή κατά την ανάπτυξη. Πιο συγκεκριμένα:

		\begin{itemize}
			\item Η σταθερή (Stable) έκδοση του κώδικα βρίσκεται στο master branch.
			\item Όποια αλλαγή πρόκειται να γίνει στον κώδικα, είτε αυτή είναι hotfix, είτε κάποιο νέο feature, είτε documentation, θα πρέπει να γίνεται αποκλειστικά σε νέο branch με χαρακτηριστικό τίτλο, ο οποίος να ξεκινά από το ανάλογο πρόθεμα (prefix). Συγκεκριμένα, αν πρόκειται για νέο feature να χρησιμοποιείται το "feature" prefix, αν πρόκειται για bugfix να χρησιμοποιείται το "bug" ή το "hotfix" prefix, και αν πρόκειται για αλλαγή στο documentation ή στο Readme, να χρησιμοποιείται το "doc" prefix. (Πχ. feature/pin\_changing)
			\item Τα νέα Branches, εφόσον ελεγχθούν, θα πρέπει να συγχωνεύονται στο αντίστοιχο branch στο οποίο απευθύνονται (βλ. παρακάτω).
			\item Νέα λειτουργικότητα (νέα features) προστίθεται μόνο στα νέα releases (στο branch του εκάστοτε νέου release). Τα hotfix καθώς επίσης και διάφορα bugs μπορούν να συγχωνεύονται απευθείας με το master, αρκεί να έχουν ελεγχθεί εξονυχιστικά πρώτα και να είναι υψηλής προτεραιότητας.
			\item Το master branch, καθώς επίσης και τα branches για νέα releases θα πρέπει να είναι κλειδωμένα και να δέχονται συγχονεύσεις μόνο εφόσον περάσει από έγκριση ο κώδικας μέσω κάποιου \idxa{Pull Request} ή \idxa{Merge Request}.
		\end{itemize} 

		Αρχικά, μέχρι την πρώτη δημόσια έκδοσή του (\verb|0.2.0|), ο κώδικας του PiLock φιλοξενούνταν στον προσωπικό εξυπηρετητή του δημιουργού, και τα αποθετήρια του διαχειρίζονταν μέσω του δωρεάν λογισμικού διαχείρησης αποθετηρίων Git γνωστό ως \idxa{GitLab} (\url{https://about.gitlab.com}). Αργότερα, από την πρώτη δημόσια έκδοσή του PiLock και μετά, ξεκίνησε να χρησιμοποιείται το \idxa{GitHub} (\url{https://github.com}) ως χώρος φιλοξενίας του έργου και των αποθετηρίων του.

	\subsection{Issue Tracking}
		Κατά την ανάπτυξη του PiLock, έγινε χρήση τόσο του συστήματος Issue Tracking του GitLab, όσο και του GitHub. Τα συστήματα \idxa{Issue Tracking}, χρησιμοποιούνται για να κρατάνε μια λίστα με διάφορα "ζητήματα" που προκύπτουν κατά την ανάπτυξη ενός έργου ή που προέρχονται από εξωτερικούς χρήστες. Στην 2η κατηγορία ανήκουν διάφορα αιτήματα νέας λειτουργικότητας, ή διάφορα bugs τα οποία μπορεί να έχουν αναφερθεί, κατά την χρήση του λογισμικού ή κατά την διάρκεια δοκιμών (testing). Issues επίσης μπορεί να προκύψουν από εξωτερικούς χρήστες ως απλά ερωτήματα για την χρήση του λογισμικού.

		Τα Issues, έχουν 2 κύριες καταστάσεις: Open (Ανοικτό), Closed (Κλειστό/Ολοκληρωμένο). Τα ανοικτά issues είναι τα αυτά που ακόμα δεν έχουν ικανοποιηθεί οι απαιτήσεις τους, ή είναι σε διαδικασία ανάπτυξης. Τα κλειστά issues είναι τα εκπληρωμένα issues ή όσα issues δεν είναι δυνατόν να εκπληρωθούν για κάποιο λόγο, και δεν πρόκειται να αναπτυχθούν άλλο.

		Προκειμένου να μπορούν να κατηγοριοποιηθούν τα Issues ενός έργου, και να τα αναλάβουν, πολλές φορές διαφορετικές ομάδες, χρησιμοποιούνται οι ετικέτες (Labels). Κάποια χαρακτηριστικά παραδείγματα χρήσης ετικετών είναι για να χωριστούν τα issues που απαιτούν νέα λειτουργικότητα από τα issues που χρησιμοποιούνται προκειμένου να επιδιορθωθεί ένα bug.

		Όπως αναφέραμε στην αρχή, κατά την ανάπτυξη του PiLock, χρησιμοποιήθηκε Issue Tracking προκειμένου να οργανωθεί περισσότερο η διαδικασία της ανάπτυξης. Αξίζει να τονιστεί οτι έπειτα από την πρώτη δημόσια έκδοση του PiLock, μπορεί ο οποιοσδήποτε χρήστης του να συνεισφέρει στην συνεισφορά νέας λειτουργικότητας ή στην αναφορά και επίλυση bugs που υπάρχουν στο σύστημα.

		Το κάθε issue καταχωρείται στο αντίστοιχο milestone. Ως "\idxa{Milestone}" (Ορόσημο) ορίζεται ένα σημαντικό σημείο κατά την ανάπτυξη του έργου. Τα Milestones δημιουργούνται από τους συντηρητές ή τους διαχειριστές ενός έργου και χρησιμοποιούνται προκειμένου να οργανωθεί καλύτερα η ανάπτυξη του έργου και για να κατηγοριοποιηθούν τα issues. Για παράδειγμα, ως milestones συνήθως ορίζονται οι νέες εκδόσεις ενός έργου, πριν να γίνουν stable. Αφότου γίνουν Stable, το milestone κλείνει.

\section{Προγραμματιστικό Περιβάλλον}
	\label{sec:ides}
	\subsection{Γλώσσες Προγραμματισμού/Markup}
		Το λογισμικό που χειρίζεται το \idxa{Business Logic} του εξυπηρετητή του PiLock ήταν, αρχικά, γραμμένο σε Python 2.7. Από την έκδοση \verb|0.3.1| και έπειτα, έγινε μετάβαση στην Python 3.6. Το γραφικό περιβάλλον είναι γραμμένο σε \idxa{HTML5 (HyperText Markup Language)}, \idxa{CSS3 (Cascading Style Sheets)} και \idxa{JavaScript}. Το αρχείο παραμετροποίησης που χρησιμοποιείται μέχρι και την τελευταία έκδοση, προκειμένου να μπορεί ο χρήστης, αν θέλει, να κλειδώσει το PiLock, είναι γραμμένο σε \idxa{YAML} (YAML Ain't Another Markup Language). Τέλος, τα σενάρια που χρησιμοποιούνται για την αρχική εγκατάσταση του PiLock στο RPi, είναι γραμμένα σε \idxa{Shell}. Το λογισμικό της εφαρμογής Android, καθώς επίσης και της εφαρμογής για Android Wear, είναι γραμμένα σε Java και \idxa{XML (Cross Markup Language)}. Τα σενάρια ξεκλειδώματος (Unlock Scripts) είναι γραμμένα σε Python και \idxa{Wiring}, ένα Framework κατασκευασμένο για προγραμματισμό μικροελεγκτών, το οποίο χρησιμοποιείται για προγραμματισμό στο Arduino.

	\subsection{Βιβλιοθήκες/Frameworks που χρησιμοποιήθηκαν}
		\label{sub:fws}
		Το Business Logic του εξυπηρετητή του PiLock είναι υλοποιημένο σε Django. Το Django είναι ένα Free And Open Source Web Framework γραμμένο σε Python και συντηρείται από το Django Software Foundation (DSF)\sucite{DSF}. Χρησιμοποιεί την αρχιτεκτονική MVT (Model-View-Template). Το Django είναι φτιαγμένο προκειμένου να καταστήσει εύκολη την ανάπτυξη πολύπλοκων ιστοσελίδων, που χρησιμοποιούν σύνδεση με βάσεις δεδομένων. Βασίζεται στην αρχή του Don't Repeat Yourself (DRY), που σημαίνει οτι είναι έτσι σχεδιασμένο, ώστε να μπορέσει να μειώσει τις επαναλήψεις κομματιών κώδικα, η την επανάληψη ορισμού λειτουργικότητας σε ένα λογισμικό\sucite{Django_Philosophies}. Το Django παρέχει σύστημα αφαιρετικότητας βάσεων δεδομένων (Database Abstraction), που σημαίνει οτι ο τελικός κώδικας είναι συμβατός με μια πληθώρα συστημάτων βάσεων δεδομένων (sqlite, MySQL, PostgreSQL). Αυτό καθιστά την μετάβαση του συστήματος (αν χρειαστεί), σε κάποιο νέο σύστημα βάσης δεδομένων εύκολη, καθώς δεν χρειάζεται να αλλάξει ο κώδικας.

		Ο πίνακας διαχείρησης του PiLock (AdminCP), χρησιμοποιεί το Bootstrap 3 framework προκειμένου να εξασφαλίσει ομαλή έμφάνιση των οπτικών στοιχείων του σε όλα τα πιθανά μεγάθη οθονών. Το Bootstrap δημιουργήθηκε από τον Mark Otto και τον Jacob Thornton, που εργάζονταν ως προγραμματιστής και designer, αντίστοιχα, στο Twitter το 2010 \textsuperscript{\cite{BS_about}}, όπου προέκυψε ανάγκη για ενοποίηση των εργαλείων που χρησιμοποιούνταν προκειμένου να είναι λιγότερο δαπανηρή η συντήρηση\sucite{BS_cr_reason}. Το Twitter είναι ένα Free and Open Source έργο, και διανείμεται υπό την άδεια MIT.
 		
 		Προκειμένου να μπορούν να εκτελεστούν κάποιες λειτουργίες του πίνακα διαχείρησης, όπως το ξεκλείδωμα απευθείας από τον πίνακα διαχείρησης, καθώς επίσης και κάποια οπτικά εφέ, χρησιμοποιήθηκε η βιβλίοθήκη jQuery. H jQuery είναι μια βιβλιοθήκη γραμμένη σε Javascript και χρησιμοποιείται προκειμένου να απλοποιήσει την διαδικασία προγραμματισμού στο επίπεδο του πελάτη (Client-Side Scripting)\sucite{JQ_about}.

 	\subsection{Προγραμματιστικά Εργαλεία που Χρησιμοποιήθηκαν}
 		Καθ' όλη την διαδικασία ανάπτυξης του PiLock, χρησιμοποιήθηκαν διάφορα \textbf{Ολοκληρωμένα Περιβάλλοντα Ανάπτυξης (Integrated Development Environments, IDE)}, τα οποία, κάποια εξ' αυτών αποτελούν λογισμικό ανοικτού κώδικα. Παρατίθενται παρακάτω.

 		\subsubsection{PyCharm - JetBrains}
 			Το PyCharm είναι ένα προγραμματιστικό περιβάλλον στοχευμένο στην ανάπτυξη λογισμικού με την γλώσσα Python. Αναπτύσσεται από την τσεχική εταιρία JetBrains\sucite{Pycharm_creator} και είναι γραμμένο σε Java και Python. Διατίθενται 2 εκδόσεις του Pycharm: Το PyCharm Community, που είναι δωρεάν και ανοικτού κώδικα λογισμικό (υπόκειται, συγκεκριμένα, στην άδεια Apache\sucite{Pycharm_comm_FOSS}), και το PyCharm Professional, το οποίο διατίθεται επι πληρωμής και είναι κλειστού κώδικα. Το PyCharm Professional παρέχει επιπρόσθετη λειτουργικότητα από αυτή του Community\sucite{Pycharm_prof_vs_comm}. Και οι δύο εκδόσεις είναι συμβατές με Windows, Linux και macOS.

 			Το PyCharm περιέχει λειτουργικότητα ανάλυσης κώδικα, γραφικό αποσφαλματωτή (graphical debugger), και ενσωμάτωση πολλών εργαλείων διαχείρησης εκδόσεων (βλ. \fullref{subsec:vc}), προκειμένου να ελαττωθεί η χρήση αυτών των προγραμμάτων εκτός του περιβάλλοντος εργασίας.

 			Το PyCharm επιλέχτηκε κατά την ανάπτυξη του PiLock Server κυρίως λόγω του γραφικού αποσφαλματωτή, της ανάλυσης κώδικα και της ενσωμάτωσης εργαλείων διαχέιρησης εκδόσεων. Από την έκδοση \verb|0.3.1| και έπειτα, χρησιμοποιήθηκε επίσης η λειτουργία Unit Testing που παρέχει προκειμένου να υλοποιηθεί μηχανισμός αυτοματοποιημένου ελέγχου εγκυρότητας κώδικα (βλ. \fullref{sec:unittesting}).

 		\subsubsection{Android Studio}
 			Για την ανάπτυξη της εφαρμογής Android και αργότερα, από την έκδοση \verb|0.3.0| και μετά, για την ανάπτυξη της εφαρμογής για Android Wear, χρησιμοποιήθηκε το Android Studio. Το \idxa{Android Studio} αποτελεί το επίσημο περιβάλλον ανάπτυξης εφαρμογών για Android συστήματα. Κατασκευάζεται από την Google σε συνεργασία με την JetBrains και είναι βασισμένο πάνω στο IntelliJ IDEA της JetBrains. Είναι γραμμένο σε Java και Kotlin.

 			Το Android Studio αποτελεί αντικαταστάτη των Εργαλείων Ανάπτυξης Android του Eclipse (Android Development Tools, ADT), τα οποία χρησιμοποιούνταν για ανάπτυξη σε Android μέχρι και το 2015\sucite{adt_deprecation}.

 		\subsubsection{Άλλα Εργαλεία, Text Editors}
 			\paragraph{Sublime Text}
 				Το \idxa{Sublime Text} είναι ένας επεξεργαστής κειμένου γραμμένος σε C++ και Python από τον Jon Skimmer και τον Will Bond. Υποστηρίζει μια πληθώρα γλωσσών προγραμματισμού και με λειτουργικότητα όπως η "Goto Anything", η οποία υποστηρίζει γρήγορη πλοήγηση σε οποιοδήποτε σημείο του κώδικα, σε διάφορα αρχεία, σύμβολα ή γραμμές επιλέγεται από πολλούς προγραμματιστές ανά τον πλανήτη.

 				Στο PiLock χρησιμοποιήθηκε κατά την ανάπτυξη μέρους του γραφικού περιβάλλοντος του πίνακα διαχείρησης συγκεκριμένα για την δημιουργία και επεξεργασία των αρχείων HTML5, CSS3 και JavaScript που τον αποτελούν.

 			\paragraph{GNU Nano}
 				Ο \idxa{GNU Nano} είναι ένας επεξεργαστής κειμένου που χρησιμοποιεί γραφική διεπαφή γραμμής εντολών (Command Line Graphical User Interface) και υπάρχει προεγκατεστημένος στα περισσότερα συστήματα είδους Unix. Είναι φτιαγμένος έτσι ώστε να μπορεί να εξομοιώνει τον Pico Editor όσο το δυνατόν καλύτερα και παράλληλα παρέχει παραπάνω λειτουργικότητα από αυτόν\sucite{nano_extra}. Είναι δωρεάν και ανοικτού κώδικα λογισμικό και υπόκειται στην άδεια GNU General Public Licence (GPL).

 				Κατά την ανάπτυξη του PiLock, το GNU Nano χρησιμοποιήθηκε προκειμένου να γίνουν δοκιμές και αποσφαλμάτωση του PiLock Server, ενόσω είναι σε λειτουργία πάνω στο Raspberry Pi, καθώς είναι προσβάσιμο από απομακρυσμένο τερματικό, μέσω SSH.

 			\paragraph{git-cola}
 				Το \idxa{git-cola} είναι ένα δωρεάν και ανοικτού κώδικα γραφικό περιβάλλον διαχείρησης αποθετηρίων Git, ανεπτυγμένο από τον David Aguilar σε Python, και χρησιμοποιεί την βιβλιοθήκη PyQt για την κατασκευή του γραφικού περιβάλλοντός του. Χρησιμοποιείται προκειμένου να καταστεί πιο εύκολη η διαχείρηση ενός αποθετηρίου git, παρέχοντας μεγάλο μέρος των λειτουργιών του git μέσω του εύχρηστου γραφικού περιβάλλοντός του. Υποστηρίζει λειτουργίες όπως ευανάγνωστη προβολή του ιστορικού του αποθετηρίου, εύχρηστη λειτουργία δημιουργίας νέων commits, και πολλές άλλες. Yπόκειται στην άδεια GNU General Public Licence (GPL), version 2\sucite{gitcola_about}.