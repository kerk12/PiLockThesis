Στον σημερινό κόσμο, οι τεχνολογικές μας ανάγκες γίνονται ολοένα και πιο πολύπλοκες. Κάθε μέρα βγαίνουν στην επιφάνεια νέες τεχνολογικές διευκολύνσεις για τον άνθρωπο, σκοπός των οποίων είναι να κάνουν την διαβίωσή του πιο "έξυπνη", δίνοντάς του τον μέγιστο έλεγχο σε κάθε σημείο της ζωής του. Με την άνθιση του internet of things, γίνεται εύκολη η διασύνδεση πολλών συσκευών (από την μικρότερη ως την μεγαλύτερη), με σκοπό τον έλεγχό τους απομακρυσμένα.

\textbf{Σκοπός της παρούσας πτυχιακής εργασίας είναι να περιγράψει την πλήρη διαδικασία του σχεδιασμού και υλοποίησης ενός συστήματος ελέγχου κλειδαριάς σπιτιού/γραφείου, γνωστό ως PiLock.}

Η εφαρμογή υλοποιήθηκε, στο μεγαλύτερο μέρος της, χρησιμοποιόντας λογισμικό τελευταίας τεχνολογίας, πράγμα που μας εγγυάται την μέγιστη ευελιξία όσων αφορά την ανάπτυξη, πράγμα που ισοδυναμεί με μέγιστη ταχύτητα ανάπτυξης και αυξημένη ασφάλεια. %Αξίζει σε αυτό το σημείο να αναφέρουμε οτι δεν πρέπει να μπερδεύουμε το λογισμικό τελευταίας τεχνολογίας με το Bleeding Edge Software (Λογισμικό τεχνολογίας αιχμής).

\section{Internet of Things}
	Ο όρος "Internet of Things"(IoT) χρησιμοποιήθηκε πρώτη φορά από τον Kevin Ashton το 1999 σε μία παρουσίασή του στην Procter \& Gamble (P\&G) \cite{iotterm}. Ο όρος επινοήθηκε προκειμένου να μπορεί να τονιστεί η δύναμη της (τότε) δημοφιλούς ιδέας της χρήσης της τεχνολογίας RFID σε συστήματα εφοδιαστικών αλυσίδων εταιριών για παρακολούθηση εμπορευμάτων. Πλέον, ο όρος Internet of Things χρησιμοποιείται προκειμένου να χαρακτηριστούν συσκευές (μικρές ή μεγάλες) με δυνατότητα σύνδεσης στο Internet. Κάποια παραδείγματα είναι τα αυτοκίνητα με ενσωματομένους αισθητήρες, τα έξυπνα σπίτια (τα οποία αποτελούνται από μια πληθώρα έξυπνων συσκευών), καθώς επίσης και συγκεκριμένες συσκευές παρακολούθησης υγείας (όπως πχ. συσκευές παρακολούθησης καρδιακού ρυθμού) με δυνατότητα σύνδεσης στο διαδίκτυο.

	Οι δυνατότητες που έχουν οι συγκεκριμένες συσκευές τις καθιστούν ικανές για σύνδεση στο internet, και κατ'επέκταση, αυξάνουν σημαντικά τις λειτουργίες τους, προσδίδοντας μεγαλύτερο έλεγχο στον χρήστη. %Check it...

\section{Αυτοματισμοί Σπιτιού - Home Automation}
	Μία από τις πιο σημαντικές υποκατηγορίες των συσκευών Internet of Things είναι οι \textbf{συσκευές αυτοματισμού σπιτιών (Home Automation Devices, Domotics \cite{domotics} )}. Οι συσκευές αυτές δίνουν στον χρήστη τους την δυνατότητα να διαχειριστεί διάφορες συσκευές του σπιτιού/γραφείου του. Οι συσκευές αυτές μπορεί να είναι συσκευές κλιματισμού, φωτισμός, συστήματα διασκέδασης (Home Theaters, Music Stereos, κτλ...), καθώς επίσης και συστήματα συναγερμού ή και διαχείρησης πρόσβασης. Το PiLock ανήκει στην τελευταία αυτή κατηγορία.

	Συνήθως, οι συσκευές αυτές συνδέονται σε ένα κεντρικό κόμβο (Hub) προκειμένου να ελέγχονται όλες από ένα μοναδικό σημείο. Η δυνατότητα αυτή μπορεί να προστεθεί σε μία επόμενη έκδοση του PiLock (βλ. μελλοντικά σχέδια). Την παρούσα χρονική στιγμή, δεν υπάρχει αυτή η δυνατότητα.

\section{Σκοπός του PiLock}
	Το PiLock ανήκει στην κατηγορία συσκευών \textbf{"έξυπνου σπιτιού" (Smart Home)}. Σκοπός του είναι να παρέχει στον χρήστη την δυνατότητα να ξεκλειδώνει εύκολα την εξώπορτα/πόρτα του σπιτιού/γραφείου του, μέσω του SmartPhone ή του SmartWatch του, όλα αυτά χρησιμοποιόντας το ασφαλέστερο δυνατόν περιβάλλον, προκειμένου να αποφευχθεί εισβολή τρίτων.

	Μέσω του \textbf{PiLock Administration Control Panel (PiLock AdminCP)}, δίνουμε στον διαχειριστή του συστήματος ένα εύχρηστο περιβάλλον διαχείρησης από το οποίο μπορεί εύκολα και γρήγορα να διαχειρίζεται το PiLock. Δίνεται δυνατότητα διαχείρησης των \textbf{εξουσιοδοτημένων χρηστών (χρήστες που μπορούν να ξεκλειδώσουν την πόρτα μέσω του PiLock)}, δυνατότητα λήψης ζωτικής σημασίας πληροφοριών για το σύστημα, καθώς επίσης και της δυνατότητας ξεκλειδώματος της πόρτας απευθείας μέσω του πίνακα διαχείρησης, χωρίς να χρειάζεται να γίνει χρήση της εφαρμογής (AdminCP Unlock).


