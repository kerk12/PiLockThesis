Ένα εκ των σημαντικότερων σημείων κατά τον σχεδιασμό του PiLock ήταν ο σχεδιασμός ενός συστήματος αυτοματοποιημένης εγκατάστασης, προκειμένου να μην χρειάζεται ο τελικός χρήστης να δαπανήσει χρόνο προκειμένου να ρυθμίσει το λογισμικό. Αυτό γίνεται μέσω δύο σεναρίων γραμμένων σε Shell, που στόχος τους είναι να εκτελέσουν σχεδόν όλα τα απαιτούμενα βήματα προκειμένου να εγκαταστηθεί και να ρυθμιστεί επιτυχώς το PiLock.

Το πρώτο σενάριο, γνωστό ως \verb|setup_apache.sh| είναι υπεύθυνο για την εγκατάσταση του Webserver υπεύθυνου για την εκτέλεση του PiLock καθώς επίσης και για την αρχική ρύθμιση του Webserver. Για Webserver χρησιμοποιείται ο Apache, μαζί με το WSGI module, απαιτούμενο για την εκτέλεση των αρχείων Python του Project. Η διαδικασία που ακολουθείται από το πρώτο σενάριο είναι η εξής:

\begin{enumerate}
	\item Εγκατάσταση του Apache, του WSGI module, της Python 3, του Python 3 PIP (υπεύθυνο για την διαχείρηση των modules που χρησιμοποιούνται από την Python), του Virtual Environment (\verb|venv|) module της Python (υπεύθυνου για την απομόνωση του περιβάλλοντως εκτέλεσης του κώδικα από το υπόλοιπο σύστημα, προκειμένου να μην τροποποιηθούν ήδη υπάρχοντα πακέτα Python 3 του συστήματος\sucite{python_venv}), καθώς επίσης και του cron (προκειμένου να καταστεί δυνατή η λειτουργία του συστήματος ειδοποιήσεων). Έπειτα από την εγκατάσταση των παραπάνω πακέτων, γίνεται ενεργοποίηση του WSGI module.
	\item Αντιγραφή των αρχείων του Project στον κατάλογο εκτέλεσης (\verb|/var/www/PiLock|) και αλλαγή ιδιοκτήτη και δικαιωμάτων στον κατάλογο και στα αρχεία αυτά. Ως ιδιοκτήτης και ομάδα ορίζονται ο \verb|www-data|, καθώς αποτελεί τον χρήστη unix που χρησιμοποιείται από τον Apache για την εκτέλεση κώδικα. Η πρόσβαση ανακαλείται από τους υπόλοιπους χρήστες και ομάδες του συστήματος (\verb|chmod 700|).
	\item Γίνεται δημιουργία venv περιβάλλοντος και εγκατάσταση των απαιτούμενων module για την λειτουργία του PiLock σε αυτό (μέσω του αρχείου \verb|requirements.txt| και του PIP).
	\item Γίνεται δημιουργία και υπογραφή του πιστοποιητικού SSL που θα χρησιμοποιείται για την διασφάλιση της επικοινωνίας μεταξύ του Server και της εφαρμογής πελάτη.
	\item Σε αυτό το σημείο πρέπει ο χρήστης να ενεργοποιήσει το πιστοποιητικό επεξεργάζοντας το αρχείο διαμόρφωσης του Apache υπεύθυνο για την χρήση του PiLock. Αυτό γίνεται επεξεργάζοντας το προεπιλεγμένο αρχείο διαμόρφωσης συνδέσεων SSL του Apache\footnote{/etc/apache2/sites-available/000\-default\-ssl.conf} και αλλάζοντας τις γραμμές \verb|SSLCertificateFile| και \verb|SSLCertificateKeyFile| με το νέο μονοπάτι στο οποίο βρίσκονται το πιστοποιητικό και το κλειδί που δημιουργήθηκε προηγουμένως (\verb|pilock.crt| και \verb|pilock.key|) αντίστοιχα. Σε μία επόμενη έκδοση του PiLock θα αυτοματοποιηθεί πλήρως αυτή η εργασία, χωρίς να χρειάζεται ο χρήστης να δαπανήσει χρόνο.
\end{enumerate}
