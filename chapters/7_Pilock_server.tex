Στο παρόν κεφάλαιο θα αναλυθεί η δομή του διακομιστή του PiLock, και παράλληλα θα εξηγηθεί ο τρόπος λειτουργίας του. Όπως είπαμε στο Κεφάλαιο \fullref{sub:fws}, ο διακομιστής του PiLock χρησιμοποιεί το Django, ένα Web Framework γραμμένο σε Python, προκειμένου να χειρίζεται το Business Logic με το οποίο λειτουργεί.

\section{Αρχιτεκτονική MTV}
	Πριν να αρχίσουμε να αναλύουμε το Business Logic του διακομιστή, είναι αναγκαίο να αναλύσουμε την Αρχιτεκτονική του Framework του οποίου χρησιμοποιούμε. Η αρχιτεκτονική του Django, γνωστή ως MTV, αποτελείται από 3 ξεχωριστά εξαρτήματα, τα οποία συνεργάζονται προκειμένου να παραχθεί η τελική σελίδα, η οποία παραδίδεται στον πελάτη:

	\begin{itemize}
		\item Το \textbf{Μοντέλο (Model, M)} είναι μία αναπαράσταση των δεδομένων τα οποία χρησιμοποιεί και επεξεργάζεται ο διακομιστής. Το μοντέλο δεν αποτελεί τα ίδια τα δεδομένα, αλλά μια διεπαφή μέσω της οποίας ο διακομιστής μπορεί να αντλήσει και να χρησιμοποιήσει δεδομένα. Για παράδειγμα, αν δημιουργούσαμε μια εφαρμογή για επαφές, η Επαφή, μαζί με όλα της τα στοιχεία (Ονοματεπώνυμο, Τηλέφωνο, κτλ...) θα αποτελούσε ένα μοντέλο. 
		\item Το \textbf{Πρότυπο (Template, T)}, το οποίο αποτελεί το στρώμα παρουσίασης (presentation layer) των δεδομένων. Καθορίζει την μορφή με την οποία θα παρουσιαστούν τα δεδομένα στον πελάτη. Καθώς το Django πρόκειται για ένα Web Framework, το Template σχηματίζει τις τελικές σελίδες Web που θα παραδώσει ο εξυπηρετητής στον πελάτη.
		\item Την \textbf{Προβολή (View, V)}, η οποία αποτελεί την επιχειρησιακή λογική (Business Logic) του εξυπηρετητή, δηλαδή τον τρόπο χειρισμού και επεξεργασίας των δεδομένων πριν να παραδοθούν στο εκάστοτε Template. Αποτελεί το στρώμα ενδιάμεσα στο Model και στο Template.
	\end{itemize}

