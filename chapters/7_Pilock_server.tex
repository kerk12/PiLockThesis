Στο παρόν κεφάλαιο θα αναλυθεί η δομή του διακομιστή του PiLock, και παράλληλα θα εξηγηθεί ο τρόπος λειτουργίας του. Όπως είπαμε στο Κεφάλαιο \fullref{sub:fws}, ο διακομιστής του PiLock χρησιμοποιεί το Django, ένα Web Framework γραμμένο σε Python, προκειμένου να χειρίζεται το Business Logic με το οποίο λειτουργεί.

\section{Αρχιτεκτονική MTV}
	\label{sec:mtv_arch}
	Πριν να αρχίσουμε να αναλύουμε το Business Logic του διακομιστή, είναι αναγκαίο να αναλύσουμε την Αρχιτεκτονική του Framework του οποίου χρησιμοποιούμε. Η αρχιτεκτονική του Django, γνωστή ως MTV, αποτελείται από 3 ξεχωριστά εξαρτήματα, τα οποία συνεργάζονται προκειμένου να παραχθεί η τελική σελίδα, η οποία παραδίδεται στον πελάτη:

	\begin{itemize}
		\item Το \textbf{Μοντέλο (Model, M)} είναι μία αναπαράσταση των δεδομένων τα οποία χρησιμοποιεί και επεξεργάζεται ο διακομιστής. Το μοντέλο δεν αποτελεί τα ίδια τα δεδομένα, αλλά μια διεπαφή μέσω της οποίας ο διακομιστής μπορεί να αντλήσει και να χρησιμοποιήσει δεδομένα. Για παράδειγμα, αν δημιουργούσαμε μια εφαρμογή για επαφές, η Επαφή, μαζί με όλα της τα στοιχεία (Ονοματεπώνυμο, Τηλέφωνο, κτλ...) θα αποτελούσε ένα μοντέλο. 
		\item Το \textbf{Πρότυπο (Template, T)}, το οποίο αποτελεί το στρώμα παρουσίασης (presentation layer) των δεδομένων. Καθορίζει την μορφή με την οποία θα παρουσιαστούν τα δεδομένα στον πελάτη. Καθώς το Django πρόκειται για ένα Web Framework, το Template σχηματίζει τις τελικές σελίδες Web που θα παραδώσει ο εξυπηρετητής στον πελάτη.
		\item Την \textbf{Προβολή (View, V)}, η οποία αποτελεί την επιχειρησιακή λογική (Business Logic) του εξυπηρετητή, δηλαδή τον τρόπο χειρισμού και επεξεργασίας των δεδομένων πριν να παραδοθούν στο εκάστοτε Template. Αποτελεί το στρώμα ενδιάμεσα στο Model και στο Template.
	\end{itemize}

\section{Django Apps}
	Ως πρώτο στρώμα για ένα οποιοδήποτε έργο στο Django αποτελεί το Project. Το Project αποτελείται από διάφορες συλλογές κώδικα γνωστές ως \textbf{Εφαρμογές (Django Apps)}, και ορίζει το περιβάλλον εκτέλεσης του κώδικα. Περιέχει όλες τις ρυθμίσεις απαραίτητες για να εκτελεστεί επιτυχώς ένα έργο. Μέσω των Apps οργανώνεται η επιχειρισιακή λογική ενός έργου σε πολλά, διακριτά κομμάτια, που είτε εξαρτόνται, είτε όχι μεταξύ τους.

	Το PiLock αποτελείται από 2 Django Apps: Την "main" (κύρια εφαρμογή), η οποία, μέσω της επιχειρισιακής λογικής τής αποτελεί την κύρια \textbf{Διεπαφή Προγραμματισμού Εφαρμογών (Application Programming Interface, API)}, υπεύθηνη για την επικοινωνία με την εφαρμογή πελάτη και για την λειτουργία του συστήματος ξεκλειδώματος, και την "AdminCP" (Administration Control Panel), η οποία αποτελεί τον πίνακα διαχείρησης του PiLock.

\section{Μοντέλα που Ορίστηκαν/Χρησιμοποιούνται}
	Όπως αναφερθήκαμε πριν (βλ. \fullref{sec:mtv_arch}), η αρχιτεκτονική του Django απαιτεί να οριστούν κάποια μοντέλα απεικόνισης των δεδομένων που θα επεξεργάζονται από τον διακομιστή. Κατά την ανάπτυξη του PiLock, ορίστηκαν τα παρακάτω μοντέλα:

	\begin{itemize}
		\item \textbf{Profile (Device Profile)}: Αποτελεί το προφίλ μιας κινητής συσκευής με εξουσιοδότηση ξεκλειδώματος. Αποτελείται από τα εξής πεδία:
		\begin{itemize}
			\item \textbf{user}: Ο χρήστης της πλατφόρμας του οποίου του ανήκει η συσκευή αυτή. "Δένεται" με έναν συγκεκριμένο χρήστη της πλατφόρμας μέσω του Πρωτεύοντος Κλειδιού του (Primary Key, PK), έτσι ώστε ο εξουσιοδοτημένος χρήστη να μπορεί να έχει αποκλειστικά μία εξουσιοδοτημένη συσκευή. Η σχέση αυτή ονομάζεται One To One.
			\item \textbf{authToken (Authorization Token)}: Ένα τυχαία παραγόμενο αλφαριθμητικό, μήκους 60 χαρακτήρων που δημιουργείται όταν δημιουργείται και το προφίλ συσκευής. Αποτελεί τεκμήριο οτι η συγκεκριμένη συσκευή είναι εξουσιοδοτημένη. Στην ΒΔ αποθηκεύεται σε Hashed μορφή (SHA512). 
			\item \textbf{pin (Personal Identification Number, PIN)}: Ένας 6ψήφιος αριθμητικός προσωπικός κωδικός. Δίδεται στον χρήστη την στιγμή που δημιουργείται το προφίλ συσκευής και μπορεί, αν θελήσει να τον αλλάξει. Στην ΒΔ αποθηκεύεται σε Hashed μορφή (SHA512). Μπορεί να είναι κενό το πεδίο, σε περίπτωση που ο χρήστης χρησιμοποιεί ξεκλείδωμα χωρίς PIN.
			\item \textbf{wearToken (Wear Token)}: Ένα τυχαία παραγόμενο αλφαρηθμιτικό, μήκους 30 χαρακτήρων που δημιουργείται την στιγμή που ο χρήστης θα συγχρονήσει το Android Wear Smartwatch του με το PiLock. Στην ΒΔ αποθηκεύεται σε Hashed μορφή (SHA512). Μπορεί να είναι κενό το πεδίο, αν ο χρήστης δεν συγχρονίσει το Smartwatch του.
		\end{itemize}
		\item \textbf{AccessAttempt (Access Attempt)}: Αποτελεί μια καταγραφή απόπειρας εισόδου/ξεκλειδόματος. Χρησιμοποιείται από το ημερολόγιο πρόσβασης του πίνακα διαχείρησης. Αποτελείται από τα εξής πεδία:
		\begin{itemize}
			\item \textbf{usenameEntered (Entered Username)}: Το όνομα χρήστη που πληκτρολόγησε ο χρήστης κατά την είσοδό του στην πλατφόρμα. Αλφαριθμητικό, μέγιστου μήκους 130 χαρακτήρων. Αν το σύστημα δεν καταφέρει να εντοπίσει κάποιο όνομα χρήστη, παραμένει κενό.
			\item \textbf{is\_unlock\_attempt}: Boolean πεδίο. Παίρνει την τιμή True αν η απόπειρα εισόδου ήταν προκειμένου να γίνει ξεκλείδωμα, False αν όχι.
			\item \textbf{successful}: Boolean πεδίο. Παίρνει την τιμή True αν έγινε επιτυχής εξακρίβωση στοιχείων και παραχωρήθη η πρόσβαση, False αν δεν παραχωρήθηκε πρόσβαση.
			\item \textbf{ip}: Διεύθυνση IP του πελάτη, κατά την απόπειρα εισόδου.
			\item \textbf{datetime}: Ημερομηνία και ώρα που πραγματοποιήθηκε η απόπειρα εισόδου στο σύστημα.
		\end{itemize}
		\item \textbf{Notification}: Αποτελεί μια ειδοποίηση που εμφανίζεται στον πίνακα διαχείρησης, προκειμένου να ειδοποιηθεί ο χρήστης για κάποιο σημαντικό ή μή ζήτημα σχετικό με την εγκατάσταση του PiLock. Αποτελείται από τα εξής πεδία:
		\begin{itemize}
			\item \textbf{type}: Ο τύπος της ειδοποίησης. Μπορεί, μέχρι και την τελευταία έκδοση (\verb|0.3.1|, κατά τον χρόνο συγγραφής της παρούσας εργασίας), να πάρει μία εκ των παρακάτω τιμών: "DEBUG", αν είναι ενεργοποιημένο το Debug Mode, "UPDATE", αν υπάρχει διαθέσιμη κάποια ενημέρωση για το PiLock, και "SEC", αν υπάρχει κάποιο ζήτημα ασφάλειας (δεν χρησιμοποιείται ακόμα, θα χρησιμοποιηθεί σε μία επόμενη έκδοση).
			\item \textbf{text}: Το κείμενο που θα εμπεριέχεται στη ειδοποίηση.
			\item \textbf{created}: Ημερομηνία και ώρα δημιουργίας της ειδοποίησης. Χρησιμοποιείται σε περίπτωση που χρειαστεί να γίνει αποσφαλμάτωση του συστήματος. 
		\end{itemize}
		Αξίζει να αναφερθεί οτι ανάλογα με τον τύπο της ειδοποίησης, και εάν το πεδίο \verb|text| παραμείνει κενό, η ειδοποίηση αυτόματα λαμβάνει ένα εκ των προκαθορισμένων κειμένων για τον συγκεκριμένο τύπο ειδοποίησης (βλ. \fullref{ch:notifications}).
	\end{itemize}

% \section{Σύστημα Εξουσιοδότησης Πρόσβασης}