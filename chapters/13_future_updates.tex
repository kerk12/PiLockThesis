Το PiLock, όπως αναφέρθηκε προηγουμένως, αποτελεί ένα έργο ανοικτού κώδικα, στο οποίο μπορούν όσοι χρήστες θέλουν, να προτείνουν ή να υλοποιήσουν νέα λειτουργικότητα ή να διορθώσουν προβλήματα στον ήδη υπάρχον κώδικα. Αξίζει όμως να αναφερθούν τρόποι επέκτασης του με νέα λειτουργικότητα η οποία μπορεί να προστεθεί στο μέλλον, σε επόμενες εκδόσεις του λογισμικού. Πέρα των παρακάτω πιθανών επεκτάσεων που θα αναλυθούν στο κεφάλαιο, είναι σημαντικό να τονιστεί οτι μπορούν να προταθούν παραπάνω επεκτάσεις λειτουργικότητας, ανοίγοντας ένα Issue στο επίσημο αποθετήριο του PiLock στο GitHub.  %TODO ADD LINK

\section{Σύνδεση με Κεντρικό Σημείο Ελέγχου Έξυπνων Συσκευών}
	Πολλές εγκαταστάσεις έξυπνων συσκευών σπιτιών παρέχουν ένα κεντρικό σημείο ελέγχου (Hub) όλων των συσκευών που είναι εγκατεστημένες μέσα στο σπίτι. Μπορεί, μέσω ενός μελλοντικού Update, να επεκταθεί το API του PiLock προκειμένου να προσφέρει λειτουργικότητα συνεργασίας με κάποιο επώνυμο ή μη Hub ελέγχου συσκευών. Για να γίνει αυτό, θα πρέπει να γίνει προσεκτική επιλογή κάποιας πλατφόρμας, αν πρόκειται να γίνει συμβατό με μόνο μία πλατφόρμα ελέγχου συσκευών, και κατόπιν να ακολουθηθεί από προσεκτικό σχεδιασμό και υλοποίηση, ή να γίνει ακόμα πιο προσεκτικός σχεδιασμός, σε περίπτωση που επιλεχθούν παραπάνω από μία πλατφόρμες ελέγχου, προκειμένου να μην υπάρξουν προβλήματα στο τελικό έργο, και για να είναι εξίσου καλό το αποτέλεσμα για όλες τις πλατφόρμες που θα στοχεύει.

	Πιθανές πλατφόρμες για να γίνει επέκταση αποτελούν το Google Home, το Apple Home, το Amazon Alexa και το OpenHAB, που αποτελεί Δωρεάν και Ανοικτού Κώδικα Λογισμικό.

\section{Ξεκλείδωμα μέσω αναγνώρισης δακτυλικού αποτυπώματος}
	Εφόσον υπάρχει ήδη μηχανισμός για ξεκλειδώματα χωρίς PIN, μπορεί να τον εκμεταλλευτεί ένα σύστημα ξεκλειδώματος που να χρησιμοποιεί το δακτυλικό αποτύπωμα του χρήστη προκειμένου να επιβεβαιώσει την ταυτότητα του χρήστη και να αποκρυπτογραφήσει το Auth Token, που χρησιμοποιείται για το ξεκλείδωμα. Το σύστημα αυτό χρησιμοποιεί τις ενσωματωμένες στα Android (από την έκδοση 6 και μετά) λειτουργίες αυθεντικοποίησης, και εγγυάται πολύ μεγαλύτερη ασφάλεια από το συμβατικό σύστημα με PIN που είναι ήδη υλοποιημένο, καθώς τα αποτυπώματα των χρηστών είναι μοναδικά και είναι δύσκολη η αντιγραφή τους\sucite{android_fingerprint_auth}.

\section{Ενσωμάτωση αυθεντικοποίησης μέσω LDAP, ADDS}
	Εφόσον κάποιος οργανισμός επιλέξει να χρησιμοποιήσει το PiLock ως σύστημα ελέγχου πρόσβασης σε διάφορα σημεία του, είναι πολλές φορές επιθυμητό να χρησιμοποιούνται ήδη υπάρχοντα συστήματα αυθεντικοποίησης όπως το \idxa{Lightweight Directory Access Protocol (LDAP)} ή το \idxa{Active Directory Directory Services (ADDS)}, που χρησιμοποιούνται από τον οργανισμό προκειμένου οι χρήστες του να αυθεντικοποιούνται σε διάφορα συστήματά του. Για να γίνει αυτό, θα πρέπει να φτιαχτεί ένα module που σκοπός του θα είναι να συνεργάζεται με αυτά τα συστήματα και να αντλεί τις ομάδες που θα του ορίζει ο διαχειριστής προκειμένου να έχουν πρόσβαση μόνο εξουσιοδοτημένες ομάδες χρηστών στο PiLock. Με αυτό τον τρόπο αυθεντικοποίησης των χρηστών αποφεύγεται η ανάγκη εκ νέου εγγραφής χρηστών ενός οργανισμού στο σύστημα του PiLock, έπειτα από την εγκατάσταση, και συγχρονίζεται αυτόματα όταν προστίθενται νέοι χρήστες.

\section{Ξεκλείδωμα Επισκεπτών}
	Προτάθηκε από τον Χαράλαμπο Νικολάτο τον Αύγουστο του 2017.\\Πολλές φορές είναι επιθυμητό να μπορεί ο χρήστης να καλέσει κάποιο φίλο ή συγγενή στο σπίτι του. Για να διευκολυνθεί η πρόσβαση του/των μπορεί να υλοποιηθεί ένα \idxa{Σύστημα Ξεκλειδώματος Προσκεκλημένων/Επισκεπτών (Guest Unlock)}. Αφότου ο διαχειριστής εισάγει την διεύθυνση ηλεκτρονικού ταχυδρομείου του επισκέπτη στο PiLock, θα δημιουργείται ένας μοναδικός κωδικός αρκετά μεγάλου μήκους (16 χαρακτήρες, πεζά, κεφαλαία γράμματα, αριθμοί) και θα αποστέλλεται στην διεύθυνση του επισκέπτη που εισήχθηκε από τον διαχειριστή. Έπειτα, θα μπορεί ο επισκέπτης, αντιγράφοντας τον κωδικό αυτόν, να τον χρησιμοποιήσει για ένα μοναδικό ξεκλείδωμα μέσω της εφαρμογής του PiLock. Έπειτα από την χρήση του, ο κωδικός θα διαγράφεται από το σύστημα και θα είναι αδύνατον να επαναχρησιμοποιηθεί.
