Το PiLock, όπως αναφέρθηκε προηγουμένως, αποτελεί ένα έργο ανοικτού κώδικα, στο οποίο μπορούν όσοι χρήστες θέλουν να προτείνουν ή να υλοποιήσουν νέα λειτουργικότητα ή να διορθώσουν προβλήματα στον ήδη υπάρχον κώδικα. Αξίζει όμως να αναφερθούν τρόποι επέκτασης του με νέα λειτουργικότητα η οποία μπορεί να προστεθεί στο μέλλον, σε επόμενες εκδόσεις του λογισμικού.

\section{Σύνδεση με Κεντρικό Σημείο Ελέγχου Έξυπνων Συσκευών}
	Πολλές εγκαταστάσεις έξυπνων συσκευών σπιτιών παρέχουν ένα κεντρικό σημείο ελέγχου (Hub) όλων των συσκευών που είναι εγκατεστημένες μέσα στο σπίτι. Μπορεί, μέσω ενός μελλοντικού Update, να επεκταθεί το API του PiLock προκειμένου να προσφέρει λειτουργικότητα συνεργασίας με κάποιο επώνυμο ή μη Hub ελέγχου συσκευών. Για να γίνει αυτό, θα πρέπει να γίνει προσεκτική επιλογή κάποιας πλατφόρμας, αν πρόκειται να γίνει συμβατό με μόνο μία πλατφόρμα ελέγχου συσκευών, και κατόπιν να ακολουθηθεί από προσεκτικό σχεδιασμό και υλοποίηση, ή να γίνει ακόμα πιο προσεκτικός σχεδιασμός, σε περίπτωση που επιλεχθούν παραπάνω από μία πλατφόρμες ελέγχου, προκειμένου να μην υπάρξουν προβλήματα στο τελικό έργο, και για να είναι εξίσου καλό το αποτέλεσμα για όλες τις πλατφόρμες που θα στοχεύει.

	Πιθανές πλατφόρμες για να γίνει επέκταση αποτελούν το Google Home, το Apple Home, το Amazon Alexa και το OpenHAB, που αποτελεί Δωρεάν και Ανοικτού Κώδικα Λογισμικό.

\section{Ξεκλείδωμα μέσω αναγνώρισης δακτυλικού αποτυπώματος}
	Εφόσον υπάρχει ήδη μηχανισμός για ξεκλειδώματα χωρίς PIN, μπορεί να τον εκμεταλλευτεί ένα σύστημα ξεκλειδώματος που να χρησιμοποιεί το δακτυλικό αποτύπωμα του χρήστη προκειμένου να επιβεβαιώσει την ταυτότητα του χρήστη και να αποκρυπτογραφήσει το Auth Token, που χρησιμοποιείται για το ξεκλείδωμα. Το σύστημα αυτό χρησιμοποιεί τις ενσωματωμένες στα Android (από την έκδοση 6 και μετά) λειτουργίες αυθεντικοποίησης, και εγγυάται πολύ μεγαλύτερη ασφάλεια από το συμβατικό σύστημα με PIN που είναι ήδη υλοποιημένο, καθώς τα αποτυπώματα των χρηστών είναι μοναδικά και είναι δύσκολη η αντιγραφή τους\sucite{android_fingerprint_auth}.