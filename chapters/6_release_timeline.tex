Το PiLock κυκλοφορεί σε συγκεκριμένες εκδόσεις. Η κάθε έκδοση μπορεί να παρέχει είτε αυξημένη λειτουργικότητα, είτε αλλαγές προκειμένου να βελτιστοποιηθεί ο ήδη υπάρχον κώδικας. Υπάρχουν 2 τύπου εκδόσεις, Minor (μικρές), Major (μεγάλες). Οι μικρές εκδόσεις συνεισφέρουν μικρές αλλαγές στην λειτουργικότητα του PiLock, και συνήθως αποτελούνται αποκλειστικά από βελτιστοποιήσεις στον κώδικα και διορθώσεις bugs. Οι μεγάλες εκδόσεις συνεισφέρουν μεγάλες αλλαγές στην λειτουργικότητα είτε του πελάτη, είτε του εξυπηρετητή και συνήθως περιέχουν πολλά bugfix.

Στις μεγάλες εκδόσεις γίνεται αλλαγή στο νούμερο ανάμεσα στην 1η και την 2η υποδιαστολή του κωδικού έκδοσης. Στις μικρές εκδόσεις αλλάζει το νούμερο μετά την 2η υποδιαστολή, από αριστερά. Συγκεκριμένα, με κάθε νέα ενημέρωση, γίνεται αύξηση κατά ένα του αντίστοιχου αριθμού. Αν είναι μεγάλη έκδοση, πρέπει να γίνει επαναφορά του αριθμού μικρής έκδοσης στο μηδέν (0).

Παρακάτω παρατίθενται οι εκδόσεις από την δημιουργία του PiLock, μέχρι τώρα:

\section{0.1.0}
	Ημερομηνίες ανάπτυξης:\\Έναρξη: 1η Απριλίου 2017\\Κυκλοφορία: 17 Μαϊου 2017

	Η έκδοση \verb|0.1.0| ήταν η 1η έκδοση που βγήκε από την πρώτη σύλληψη του PiLock και μετά και παρείχε την βασική λειτουργικότητα, αρκετή για να λειτουργήσει το κύκλωμα ξεκλειδώματος. 

	Μέχρι τότε, η εφαρμογή πελάτη για Android είχε πολλά προβλήματα κατά την λειτουργία της, με πολλά crash, αλλά το βασικό σύστημα ξεκλειδώματος είχε υλοποιηθεί και δούλευε.

\section{0.2.0}
	Ημερομηνίες ανάπτυξης:\\Έναρξη: 29 Ιουλίου 2017\\Κυκλοφορία: 18 Αυγούστου 2017

	Η έκδοση \verb|0.2.0| αποτελεί την πρώτη δημόσια διαθέσιμη έκδοση του PiLock. Σε αυτή την έκδοση έγινε η προσθήκη του πίνακα διαχείρησης στην πρωτογενή μορφή του (πριν την κυκλοφορία της έκδοσης αυτής, οι χρήστες έπρεπε να χρησιμοποιήσουν τον συμπεριλαμβανόμενο στο Django πίνακα διαχείρησης).

	Επίσης, έγινε μια τεράστια αλλαγή στον οπτικό σχεδιασμό της εφαρμογής. Το κύριο χρώμα της εφαρμογής mobile άλλαξε σε ρόζ (\verb|#790022|) και έγινε προσθήκη ενός νέου λογότυπου φτιαγμένο από τον Δημήτρη Τζιλιβάκη.

	Στον πίνακα διαχείρησης προστέθηκε δυνατότητα διαχείρησης χρηστών, συγκεκριμένα, λειτουργία προσθήκης νέων και διαγραφής ήδη υπάρχοντων χρηστών και προφίλ συσκευών, καθώς επίσης και ένα ημερολόγιο ξεκλειδωμάτων και συνδέσεων στο σύστημα, για λόγους ασφάλειας.  

\section{0.3.0}
	Ημερομηνίες ανάπτυξης:\\Έναρξη: 7 Σεπτεμβρίου 2017\\Κυκλοφορία: 11 Νοεμβρίου 2017

	Στην έκδοση \verb|0.3.0| προστέθηκε πλήθος νέων λειτουργιών που βοηθούν στην διαχείρηση και στην χρήση του PiLock. Προστέθηκε, συγκεκριμένα δυνατότητα ξεκλειδώματος χωρίς την χρήση PIN, ένα νέο σύστημα ειδοποιήσεων προσβάσιμο από τον πίνακα διαχείρησης από όπου ο χρήστης μπορεί να δει διάφορα μηνύματα σχετικά με την κατάσταση του PiLock, καθώς επίσης και δυνατότητα ξεκλειδώματος μέσω του πίνακα διαχείρησης.

	Στην νέα λειτουργικότητα εντάσσεται επίσης η εφαρμογή για Android Wear συσκευές, που επιτρέπει στον χρήστη, εφόσον είναι συνδεδεμένο το SmartPhone του με το Smartwatch του, να γίνει ξεκλείδωμα, μέσω του Android Wear Smartwatch του.

	Τέλος, προστέθηκε hashing στην βάση δεδομένων προκειμένου να αποτρέψει προβολή των κλειδιών πρόσβασης σε περίπτωση που υπάρξει μη εξουσιοδοτημένη πρόσβαση στην βάση δεδομένων\sucite{pw_hashing} (βλ \fullref{sec:sec}). 

\section{0.3.1}
	Ημερομηνίες ανάπτυξης:\\Έναρξη: 25 Φεβρουαρίου 2018\\Κυκλοφορία: 4 Μαρτίου 2018

	Στην Minor έκδοση \verb|0.3.1| του PiLock, έγινε ενημέρωση του κώδικα ώστε να χρησιμοποιείται πλέον η Python 3.6, καθώς η Python 2.7 φτάνει σιγά σιγά στο τέλος ζωής της\sucite{Py27_EOL} και οι νέες ενημερώσεις για το Django θα γίνονται μόνο στην έκδοση 3 της Python\sucite{Django_NewUpdates}. Έγιναν πολλές διορθώσεις στον κώδικα του Server, προκειμένου να λειτουργεί καλύτερα, ενεργοποιήθηκε η χρήση του venv και ενεργοποιήθηκε η δυνατότητα αυτοματοποιημένων ελέγχων εγκυρότητας κώδικα μέσω Continuous Integration (CI).

	Από άποψης λειτουργικότητας, έγινε προσθήκη σελιδοποίησης στο ημερολόγιο πρόσβασης του συστήματος, προκειμένου να φαίνονται οι απόπειρες πρόσβασης πιο καθαρά.