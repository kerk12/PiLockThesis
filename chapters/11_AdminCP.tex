Στην πρώτη δημόσια έκδοση του PiLock (\verb|0.2.0|), έγινε η προσθήκη του πίνακα διαχείρησης του PiLock, γνωστού ως AdminCP (Administration Control Panel). Πριν να γίνει η προσθήκη αυτού, οι χρήστες υποχρεούνταν να χρησιμοποιούν τον ενσωματωμένου πίνακα διαχείρησης που παρέχει το Django, για την διαχείρηση των χρηστών και τον μοντέλων του συστήματος.

Στον πυρήνα του, ο πίνακας διαχείρησης, προσβάσιμος στο \verb|https://<PiLock_Root>/AdminCP|, όπου PiLock\_Root η διεύθυνση της εγκατάστασης του PiLock, περιέχει τις πιο αναγκαίες λειτουργίες για την διαχείρηση της πλατφόρμας. Αυτές είναι:

\begin{itemize}
	\item Προσθήκη, διαγραφή χρηστών. Οι χρήστες μπορούν να οριστούν ως λειτουργικό προσωπικό (Staff), προκειμένου να μπορούν να έχουν πρόσβαση στον πίνακα διαχείρησης. Όσοι χρήστες δεν έχουν οριστεί ως Staff, δεν μπορούν να έχουν πρόσβαση στον πίνακα διαχείρησης του PiLock, αλλά μπορούν να πραγματοποιήσουν ξεκλείδωμα.
	\item Διαγραφή υπάρχοντων προφίλ συσκευών. Σε περίπτωση που κάποιος χρήστης επιθυμεί, μπορεί να ζητήσει από τον διαχειριστή του συστήματος να διαγράψει το προφίλ της συσκευής του, προκειμένου να ξανακάνει σύνδεση στο σύστημα του PiLock, σε περίπτωση που χαθεί η συσκευή του ή για κάποιο λόγο διαγραφεί το τεκμήριο πρόσβασης από την συσκευή του.
	\item Προβολή του ιστορικού πρόσβασης. Θα αναλυθεί εκτενώς παρακάτω.
	\item Πραγματοποίηση ξεκλειδώματος απευθείας από τον πίνακα διαχείρησης. Κάποιες φορές, είναι επιθυμητό να γίνεται ξεκλείδωμα απευθείας από τον πίνακα διαχείρησης του PiLock (για λόγους ευελιξίας, ή σε περίπτωση που κάποιος διαχειριστής δεν έχει εγκατεστημένη την εφαρμογή στο κινητό του), για αυτό τον λόγο έγινε προσθήκη του μηχανισμού αυτού.  %TODO Might wanna fix this...
	\item Σύστημα ειδοποιήσεων. Προβάλλει ειδοποιήσεις σχετικές με την υγεία του συστήματος. Θα αναλυθεί εκτενώς παρακάτω.
\end{itemize}

\section{Ημερολόγιο Πρόσβασης - Access Log}
	Κάποιες φορές, ειδικά σε περιπτώσεις που το PiLock χρησιμοποιείται από πολλούς χρήστες, ή σε περίπτωση που χρησιμοποιηθεί σε κάποιο εταιρικό περιβάλλον, είναι θεμιτό να κρατείται αρχείο με το ιστορικό οποιασδήποτε πρόσβασης στο σύστημα, για λόγους ασφαλείας. Στην έκδοση \verb|0.2.0|, δημιουργήθηκε ένα τέτοιο σύστημα. Συγκεκριμένα, χρησιμοποιόντας το \verb|AccessAttempt| μοντέλο (βλ. \fullref{sec:models}), κάθε φορά που γίνεται μια απόπειρα σύνδεσης ή ξεκλειδώματος, δημιουργείται ένα νέο αντικείμενο του \verb|AccessAttempt|, και δίνονται σε αυτό κάποια από τα δεδομένα σχετικά με τον χρήστη που πραγματοποίησε την απόπειρα αυτή, δηλαδή το όνομα χρήστη του, η διεύθυνση IP του, εάν είναι απόπειρα ξεκλειδώματος, εάν η απόπειρα είναι επιτυχής, καθώς επίσης και η ημερομηνία και η ώρα της απόπειρας.

	Τα στοιχεία αυτά μπορούν να χρησιμοποιηθούν από τον διαχειριστή της εγκατάστασης προκειμένου να διασταυρωθούν με κάποιο γεγονός σχετικό με την ασφάλεια του κτιρίου της εγκατάστασης (σε περίπτωση που γίνει κάποια επίθεση/ληστεία στο κτήριο από κάποιον κακόβουλο εξουσιοδοτημένο χρήστη), ή για να εξιχνιαστούν επιθέσεις επανειλημμένων αποπειρών ξεκλειδώματος ή/και επιθέσεων άρνησης υπηρεσίας (Denial of Service Attacks).

\section{Σύστημα ειδοποιήσεων}
	Στην έκδοση \verb|0.3.0| έγινε προσθήκη στον πίνακα διαχείρησης ένα σύστημα ειδοποιήσεων. Οι ειδοποιήσεις αυτές στόχο έχουν να ειδοποιούν τον χρήστη όποτε υπάρχει κάποιο συμβάν που μπορεί να κλονίσει την ασφάλεια του συστήματος. Υπάρχουν (μέχρι την έκδοση \verb|0.3.1|) τρία είδη ειδοποιήσεων:

	\begin{itemize}
		\item \textbf{Debug Mode}: Ενεργοποιείται όταν το PiLock τρέχει με ενεργοποιημένη την λειτουργία Debug. Κατά την λειτουργία αυτή, δεν πραγματοποιούνται ξεκλειδώματα, και το σύστημα χρησιμοποιείται μόνο για αποσφαλμάτωση (κατά την ανάπτυξη νέων λειτουργιών ή την διόρθωση ήδη υπάρχοντων). Χαρακτηριστικό αυτής της λειτουργίας είναι η εμφάνιση όλων των μεταβλητών κατά την στιγμή εκτέλεσης σε περίπτωση που υπάρξει κάποιο σφάλμα. Για αυτό τον λόγο αυτόματα απενεργοποιείται ο μηχανισμός ξεκλειδωμάτων ενόσω η λειτουργία αποσφαλμάτωσης είναι ενεργή, καθώς μπορεί να γίνει διαρροή ευαίσθητων δεδομένων.
		\item \textbf{Update}: Ενεργοποιείται όποτε υπάρχει διαθέσιμη κάποια ενημέρωση για το PiLock. Ο έλεγχος γίνεται με χρήση ενός Cron Job, προγραμματισμένου να εκτελείται κάθε 60 λεπτά, χρησιμοποιόντας το Crontab του συστήματος, καθώς επίσης και το \verb|django-cron|\footnote{\url{https://github.com/Tivix/django-cron}} module προκειμένου να γίνει εύκολη ενσωμάτωση με το λογισμικό του Server. Ο έλεγχος γίνεται συγκρίνοντας το commit hash του Server που υπάρχει εγκαταστημένος, με το τελευταίο commit hash που υπάρχει στο master branch στο αποθετήριο του λογισμικού στο GitHub, καλώντας το API του GitHub.
		\item \textbf{Security}: Δεν έχει υλοποιηθεί ακόμα. Θα χρησιμοποιείται προκειμένου να ειδοποιήσει για κάποιο θέμα ασφαλείας σχετικό με την πλατφόρμα, για παράδειγμα άν κάποιος χρήστης πραγματοποιεί συνεχόμενα αποτυχημένες απόπειρες ξεκλειδώματος, ή αν είναι αποσυνδεδεμένο το Arduino από το Raspberry Pi.
	\end{itemize}

	Οι ειδοποιήσεις εμφανίζονται στην κεντρική σελίδα του πίνακα διαχείρησης, με την μορφή Bootstrap Alerts.

%TODO Might wanna write more...