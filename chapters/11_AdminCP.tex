Στην πρώτη δημόσια έκδοση του PiLock (\verb|0.2.0|), έγινε η προσθήκη του πίνακα διαχείρησης του PiLock, γνωστού ως AdminCP (Administration Control Panel). Πριν να γίνει η προσθήκη αυτού, οι χρήστες υποχρεούνταν να χρησιμοποιούν τον ενσωματωμένου πίνακα διαχείρησης που παρέχει το Django, για την διαχείρηση των χρηστών και τον μοντέλων του συστήματος.

Στον πυρήνα του, ο πίνακας διαχείρησης, προσβάσιμος στο \verb|https://<PiLock_Root>/AdminCP|, όπου PiLock\_Root η διεύθυνση της εγκατάστασης του PiLock, περιέχει τις πιο αναγκαίες λειτουργίες για την διαχείρηση της πλατφόρμας. Αυτές είναι:

\begin{itemize}
	\item Προσθήκη, διαγραφή χρηστών. Οι χρήστες μπορούν να οριστούν ως λειτουργικό προσωπικό (Staff), προκειμένου να μπορούν να έχουν πρόσβαση στον πίνακα διαχείρησης. Όσοι χρήστες δεν έχουν οριστεί ως Staff, δεν μπορούν να έχουν πρόσβαση στον πίνακα διαχείρησης του PiLock, αλλά μπορούν να πραγματοποιήσουν ξεκλείδωμα.
	\item Διαγραφή υπάρχοντων προφίλ συσκευών. Σε περίπτωση που κάποιος χρήστης επιθυμεί, μπορεί να ζητήσει από τον διαχειριστή του συστήματος να διαγράψει το προφίλ της συσκευής του, προκειμένου να ξανακάνει σύνδεση στο σύστημα του PiLock, σε περίπτωση που χαθεί η συσκευή του ή για κάποιο λόγο διαγραφεί το τεκμήριο πρόσβασης από την συσκευή του.
	\item Προβολή του ιστορικού πρόσβασης. Θα αναλυθεί εκτενώς παρακάτω.
	\item Πραγματοποίηση ξεκλειδώματος απευθείας από τον πίνακα διαχείρησης. Κάποιες φορές, είναι επιθυμητό να γίνεται ξεκλείδωμα απευθείας από τον πίνακα διαχείρησης του PiLock (για λόγους ευελιξίας, ή σε περίπτωση που κάποιος διαχειριστής δεν έχει εγκατεστημένη την εφαρμογή στο κινητό του), για αυτό τον λόγο έγινε προσθήκη του μηχανισμού αυτού.  %TODO Might wanna fix this...
	\item Σύστημα ειδοποιήσεων. Προβάλλει ειδοποιήσεις σχετικές με την υγεία του συστήματος. Θα αναλυθεί εκτενώς παρακάτω.
\end{itemize}