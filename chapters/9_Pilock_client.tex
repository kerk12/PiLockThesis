Η εφαρμογή πελάτη, γνωστή ως PiLock Client είναι μια Smartphone εφαρμογή (Smartphone App), κατασκευασμένη για Android Smartphones. Μπορεί να λειτουργήσει σε οποιαδήποτε συσκευή που διαθέτει έκδοση Android Kitkat (API Level 19) ή ανώτερη. Σύμφωνα με την διανομή αγοράς εκδόσεων Android\sucite{android_api_distribution} για την συγκεκριμένη έκδοση Android, καλύπτεται το 90\% των συσκευών που βρίσκονται αυτή την στιγμή στην αγορά, οπότε είναι ιδανική η συγκεκριμένη επιλογή έκδοσης. Η εφαρμογή Android είναι εξ'ολοκλήρου κατασκευασμένη στο Android Studio (βλ \fullref{sec:ides}), και χρησιμοποιεί για λογισμικό διαχείρησης εκδόσεων, όπως και το λογισμικό του εξυπηρετητή, το Git.

\section{Activities}
	Η εφαρμογή αποτελείται από διάφορα Activities. Ώς Activity ορίζεται μία μοναδική, στοχευμένη γραφική διεπαφή χρήστη από την οποία ο χρήστης μπορεί να συγκεντρώνεται σε μία συγκεκριμένη κατάσταση\sucite{android_ref_activity}. Στην τελευταία έκδοση του PiLock (\verb|0.3.1|), υπάρχουν συνολικά τέσσερα (4) Activities:

	\begin{itemize}
		\item \textbf{LoginActivity}: Το Activity υπεύθυνο για την εισαγωγή στοιχείων και σύνδεση του χρήστη στο σύστημα του PiLock. Μέσω αυτού υλοποιείται το 1ο στάδιο εξουσιοδότησης. Αποτελείται από 2 πεδία κειμένου που δέχονται το Username και το Password του χρήστη καθώς επίσης και ένα πλήκτρο, μέσω του οποίου γίνεται αποστολή των στοιχείων αυτών στον Server.
		\item \textbf{PINEntryActivity}: Activity, μέσω του οποίου ο χρήστης εισάγει το PIN του και στέλνει αιτήματα ξεκλειδώματος. Μέσω αυτού υλοποιείται το 2ο στάδιο εξουσιοδότητης. Αποτελείται από ένα πεδίο κειμένου στο οποίο ο χρήστης μπορεί να εισάγει το PIN του καθώς επίσης και ένα πλήκτρο με το οποίο γίνεται αποστολή των στοιχείων αυτών στον Server.
		\item \textbf{ChangePinActivity}: Υπεύθυνο για την αλλαγή PIN του χρήστη. Αποτελείται από 3 πεδία κειμένου στα οποία ο χρήστης μπορεί να εισάγει το παλιό του PIN, το νέο του PIN καθώς επίσης και το νέο του PIN για δεύτερη φορά για επιβεβαίωση. Τέλος, αποτελείται και από ένα πλήκτρο με το οποίο γίνεται αποστολή του αιτήματος αλλαγής PIN στον Server.
		\item \textbf{SettingsActivity}: Μαζί με το \verb|SettingsFragment|, αποτελούν την οθόνη "ρυθμίσεων" της εφαρμογής. Μέχρι και την τελευταία έκδοση της εφαρμογής, η οθόνη ρυθμίσεων χρησιμοποιείται προκειμένου να οριστεί η διεύθυνση του Server, κατά την πρώτη εκτέλεση της εφαρμογής. Σε μία επόμενη έκδοση της εφαρμογής, θα προστεθεί επιλογή να ορίζεται ο Server ως προσβάσιμος αποκλειστικά μέσω WiFi (βλ. \fullref{ch:future_expansion}).
	\end{itemize}

\section{Αιτήματα HTTPS}
	Όπως αναφέρθηκε προηγουμένως, όλες οι ανταλλαγές πληροφοριών ανάμεσα στην εφαρμογή και τον Server γίνονται χρησιμοποιόντας το πρωτόκολλο HTTPS. Προκειμένου να καταστεί εύχρηστο, δημιουργήθηκε μια βιβλιοθήκη αποτελούμενη από 3 κλάσεις και ένα enum, η οποία αναλαμβάνει την αποστολή και λήψη πληροφοριών προς και από τον Server.

	Η κύρια κλάση-γονέας (Superclass) λέγεται \verb|HttpsRequest|, και χρησιμοποιείται προκειμένου να κρατά τις πληροφορίες ενός οποιουδήποτε HTTPS αιτήματος, όπως για παράδειγμα την διεύθυνση του αποδοχέα, τις παραμέτρους, καθώς επίσης και τις πληροφορίες που λαμβάνονται αφότου εκτελεστεί το αίτημα, όπως τον κωδικό απάντησης, καθώς επίσης και την ίδια την απάντηση. Μέσα σε αυτή την κλάση ορίστηκε η εσωτερική κλάση \verb|RequestNotExecutedException|, η οποία αποτελεί κλάση εξαίρεσης και χρησιμοποιείται σε περίπτωση που ο χρήστης ζητήσει να λάβει τα δεδομένα της απάντησης πρωτού εκτελεστεί το αίτημα. Τέλος, ορίζεται και το \verb|HttpsRequestListener| interface, το οποίο καλείται όταν ολοκληρωθεί ένα αίτημα. Αυτό καθιστά εύκολη την άμεση αυτοματοποίηση διαδικασιών με την λήξη των αιτημάτων.	Η κλάση \verb|HttpsRequest| κληρονομείται από δύο κλάσεις γνωστές ως \verb|HttpsGET| και \verb|HttpsPOST|, υπεύθυνες για αποστολή αιτημάτων GET και POST, αντίστοιχα. 

	Η διαδικασία αποστολής των αιτημάτων αυτών είναι η εξής:

	\begin{enumerate}
		\item Δημιουργία αντικειμένου τύπου \verb|HttpsGET| ή \verb|HttpsPOST|. Στον κατασκευαστή (Constructor), δίνονται ως ορίσματα η διεύθυνση (URL) του παραλήπτη και (προαιρετικά) οι παράμετροι του αιτήματος, σε μορφή \verb|HashMap<String, String>|.
		\item Σε περίπτωση που χρειαστεί να εκτελεστεί κάποια ενέργεια με την λήξη του αιτήματος, πρέπει να γίνει ορισμός του Request Listener μέσω της μεθόδου \verb|setRequestListener(HttpsRequest.HttpsRequestListener listener)|. Στο σώμα του HttpRequestListener, θα πρέπει να γίνει υλοποίηση της μεθόδου \verb|onRequestCompleted()|. Μόλις λήξει το αίτημα, και κλείσει η σύνδεση, θα εκτελεστεί αυτή η μέθοδος.
		\item Κλήση της μεθόδου \verb|SendGET(Context c)| ή \verb|SendPOST(Context c)|, ανάλογα τον τύπο του αιτήματος.
		\item Μετά την ολοκλήρωση του αιτήματος, μπορούμε να κάνουμε ανάκτηση του κωδικού απόκρισης καθώς επίσης και την ίδια την απόκριση καλόντας την μέθοδο \verb|getResponseCode()| και \verb|getResponse()|. Είναι καλό να γίνεται έλεγχος του αιτήματος για σφάλμα χρησιμοποιόντας την μέθοδο \verb|hasError()|, ανάκτηση του σε περίπτωση που υπάρχει μέσω της μεθόδου \verb|getError()| και να γίνεται σύγκριση με τις τιμές του \verb|HttpsConnectionError| για να διαπιστωθεί ποιο είναι το σφάλμα. Να σημειωθεί οτι το enum \verb|HttpsConnectionError| συμβολίζει μόνο σφάλματα σύνδεσης, και όχι σφάλματα μέσα στην ίδια την απόκριση. Για να γίνει διάγνωση της ίδιας της απόκρισης καλό είναι να γίνεται ανάκτηση και σύγκριση του κωδικού απάντησης με τους ήδη γνωστούς κωδικούς απάντησης του πρωτόκολλου HTTP (HTTP Response Codes).
		\item Αφότου γίνει ανάκτηση της απόκρισης, μπορεί να αναλυθεί μέσω της βιβλιοθήκης αποκωδικοποίησης JSON της Java.
	\end{enumerate}
