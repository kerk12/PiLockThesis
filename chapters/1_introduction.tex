Στον σημερινό κόσμο, οι τεχνολογικές μας ανάγκες γίνονται ολοένα και πιο πολύπλοκες. Κάθε μέρα βγαίνουν στην επιφάνεια νέες τεχνολογικές διευκολύνσεις για τον άνθρωπο, σκοπός των οποίων είναι να κάνουν την διαβίωσή του πιο "έξυπνη", δίνοντάς του τον μέγιστο έλεγχο σε κάθε σημείο της ζωής του. Με την άνθιση του internet of things, γίνεται εύκολη η διασύνδεση πολλών συσκευών (από την μικρότερη ως την μεγαλύτερη), με σκοπό τον έλεγχό τους και την άντληση δεδομένων από αυτές, εξ'αποστάσεως.

\textbf{Σκοπός της παρούσας πτυχιακής εργασίας είναι να περιγράψει την πλήρη διαδικασία του σχεδιασμού και υλοποίησης ενός συστήματος ελέγχου κλειδαριάς σπιτιού/γραφείου, γνωστό ως PiLock.}

Η εφαρμογή υλοποιήθηκε, στο μεγαλύτερο μέρος της, χρησιμοποιώντας λογισμικό τελευταίας τεχνολογίας, πράγμα που μας εγγυάται την μέγιστη ευελιξία όσων αφορά την ανάπτυξη, και ισοδυναμεί με μέγιστη ταχύτητα ανάπτυξης και αυξημένη ασφάλεια. %Αξίζει σε αυτό το σημείο να αναφέρουμε οτι δεν πρέπει να μπερδεύουμε το λογισμικό τελευταίας τεχνολογίας με το Bleeding Edge Software (Λογισμικό τεχνολογίας αιχμής).

\section{Internet of Things}
	Ο όρος "\idxa{Internet of Things}" (IoT) χρησιμοποιήθηκε πρώτη φορά από τον Kevin Ashton το 1999 σε μία παρουσίασή του στην Procter \& Gamble (P\&G) \sucite{iotterm}. Επινοήθηκε προκειμένου να μπορεί να τονιστεί η δύναμη της (τότε) δημοφιλούς ιδέας της χρήσης της τεχνολογίας RFID σε συστήματα εφοδιαστικών αλυσίδων εταιριών για παρακολούθηση εμπορευμάτων. Σήμερα, ο όρος Internet of Things χρησιμοποιείται προκειμένου να χαρακτηριστούν συσκευές (μικρές ή μεγάλες) με δυνατότητα σύνδεσης στο Internet. Κάποια παραδείγματα είναι τα αυτοκίνητα με ενσωματομένους αισθητήρες, τα έξυπνα σπίτια (τα οποία αποτελούνται από μια πληθώρα έξυπνων συσκευών), καθώς επίσης και συγκεκριμένες συσκευές παρακολούθησης υγείας (όπως πχ. συσκευές παρακολούθησης καρδιακού ρυθμού) με δυνατότητα σύνδεσης στο διαδίκτυο.

	Οι δυνατότητες που έχουν οι συγκεκριμένες συσκευές τις καθιστούν ικανές για σύνδεση στο internet, και κατ'επέκταση, αυξάνουν σημαντικά τις λειτουργίες τους, προσδίδοντας μεγαλύτερο έλεγχο στον χρήστη. %Check it...

\section{Αυτοματισμοί Σπιτιού - Home Automation}
	Μία από τις πιο σημαντικές υποκατηγορίες των συσκευών Internet of Things είναι οι \textbf{συσκευές αυτοματισμού σπιτιών (Home Automation Devices, \idxa{Domotics} \sucite{domotics} )}. Οι συσκευές αυτές δίνουν στον χρήστη τους την δυνατότητα να διαχειριστεί διάφορες συσκευές του σπιτιού/γραφείου του εξ'αποστάσεως. Αυτές μπορεί να είναι συσκευές κλιματισμού, φωτισμός, συστήματα διασκέδασης (Home Theaters, Music Stereos, κτλ...), καθώς επίσης και συστήματα συναγερμού ή και διαχείρησης πρόσβασης. Το PiLock ανήκει στην τελευταία αυτή κατηγορία.

	Συνήθως, οι συσκευές αυτές συνδέονται σε ένα κεντρικό κόμβο (Hub) προκειμένου να ελέγχονται όλες από ένα μοναδικό σημείο. Η δυνατότητα αυτή μπορεί να προστεθεί σε μία επόμενη έκδοση του PiLock (βλ. \fullref{ch:future_expansion}). Την παρούσα χρονική στιγμή, δεν υπάρχει αυτή η δυνατότητα.

\section{Σκοπός του PiLock}
	Το PiLock ανήκει στην κατηγορία συσκευών \textbf{"έξυπνου σπιτιού" (Smart Home)}. Σκοπός του είναι να παρέχει στον χρήστη την δυνατότητα να ξεκλειδώνει εύκολα την εξώπορτα/πόρτα του σπιτιού/γραφείου του, μέσω του SmartPhone ή του SmartWatch του. Όλα αυτά χρησιμοποιώντας το ασφαλέστερο δυνατόν περιβάλλον, προκειμένου να αποφευχθεί εισβολή τρίτων.

	Μέσω του \textbf{\idxa{PiLock Administration Control Panel (PiLock AdminCP)}}, δίνεται στον διαχειριστή του συστήματος ένα εύχρηστο περιβάλλον διαχείρησης από το οποίο μπορεί εύκολα και γρήγορα να διαχειρίζεται το PiLock. Δίνεται δυνατότητα διαχείρησης των \textbf{εξουσιοδοτημένων χρηστών (χρήστες που μπορούν να ξεκλειδώσουν την πόρτα μέσω του PiLock)}, δυνατότητα λήψης ζωτικής σημασίας πληροφοριών για το σύστημα, καθώς επίσης και της δυνατότητας ξεκλειδώματος της πόρτας απευθείας μέσω του πίνακα διαχείρησης, χωρίς να χρειάζεται να γίνει χρήση της εφαρμογής (AdminCP Unlock).

	Ένας από τους στόχους, κατά τον σχεδιασμό του PiLock ήταν η διατήρηση του κόστους στο χαμηλότερο δυνατόν. Για να επιτευχθεί ο στόχος αυτός, χρησιμοποιήθηκε αυστηρά δωρεάν παρεχόμενο λογισμικό ανοικτού κώδικα, καθώς επίσης και εξαρτήματα εύκολα προσκομίσιμα (βλ. \fullref{ch:structure}).

\section{Έρευνα αγοράς - Καινοτομία του PiLock}
	Έπειτα από έρευνα που έγινε πάνω σε ήδη υπάρχοντες μηχανισμούς ξεκλειδώματος μέσω Raspberry Pi βρέθηκε οτι το PiLock είναι το πρώτο σύστημα ξεκλειδώματος που συνδέεται απευθείας πάνω στο κύκλωμα του θυροτηλεφώνου και χειρίζεται την κλειδαριά. 

	\subsection{Μη Εμπορικές/DIY Εφαρμογές}
		Ανάμεσα στα συστήματα που βρέθηκαν, υπάρχει ένα σύστημα που συνδέεται απευθείας επάνω στην κλειδαριά της πόρτας, αλλά σε κλειδαριά διαφορετικού τύπου από ό,τι χρησιμοποιείται στις περισσότερες πολυκατοικίες στην Ελλάδα, δημιουργημένο από έναν YouTuber γνωστό ως Hacker Shack\footnote{https://bit.ly/2LGTSJd}. Το συγκεκριμένο σύστημα χρησιμοποιείται σε κλειδαριές τύπου Deadbolt, αντί για κλειδαριές τύπου Electric Strike (βλ. \fullref{ch:unlock_mechanism}).

		Υπάρχει ένα παρόμοιο, επίσης, σύστημα με την ονομασία "Pi-Lock", κατασκευασμένο από τον Paolo Bernasconi\footnote{http://www.pi-lock.com/}, το οποίο χρησιμοποιεί Raspberry Pi αλλά προσφέρει λειτουργικότητα ξεκλειδώματος μέσω RFID, έναντι των ξεκλειδωμάτων μέσω Android, τα οποία προσφέρει το PiLock.

	\subsection{Εμπορικές Εφαρμογές}
		Έπειτα από έρευνα που έγινε στις υπάρχουσες εμπορικές εφαρμογές έξυπνης κλειδαριάς, που μπορεί να προμηθευτεί ο οποιοσδήποτε, εξάγεται το συμπέρασμα οτι η πλειονότητα αυτών των εφαρμογών (παίρνοντας ως δείγμα το άρθρο με τις καλύτερες έξυπνες κλειδαριές του 2018, από το PC Magazine)\sucite{best_sl} απαιτούν να ξοδευτεί αρκετά μεγάλο χρηματικό ποσό, σε σύγκριση με το ποσό που πρέπει να ξοδευτεί για να αγοραστούν τα εξαρτήματα του PiLock, και συνήθως απαιτείται αντικατάσταση της ήδη υπάρχουσας κλειδαριάς, πράγμα που σημαίνει οτι μπορεί να χρειαστούν παραπάνω χρήματα για την πρόσληψη τεχνικού που θα πραγματοποιήσει την αντικατάσταση.

	\subsection{Διαφορές του PiLock με τις ήδη υπάρχουσες εφαρμογές}
		\label{pilock_innovation}
		Το PiLock, εκτός του οτι είναι πολύ φθηνότερο σε σχέση με τις ήδη υπάρχουσες εμπορικές εφαρμογές που κυκλοφορούν, είναι πολύ ευκολότερο στην εγκατάσταση και μπορεί να εγκατασταθεί απευθείας στο ήδη υπάρχον σύστημα θυροτηλεφώνου που έχουν οι πολυκατοικίες, χωρίς να χρειαστεί να γίνει αλλαγή κλειδαριάς. Πέραν αυτού, με την προσθήκη συμβατότητας με Android Wear που έγινε στην έκδοση \verb|0.3.0|, είναι μία από τις πρώτες εφαρμογές παγκοσμίως που υποστηρίζουν ξεκλείδωμα πόρτας μέσω Smartwatch, και ίσως μία από τις εξελιγμένες, καθώς τα Project που υπάρχουν είναι σε πρωτογενή μορφή και δεν παρέχουν το πλήρες περιβάλλον διαχείρησης που παρέχει το PiLock.